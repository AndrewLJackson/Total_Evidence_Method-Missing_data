\documentclass[11pt]{letter}
\usepackage[a4paper,left=2.5cm, right=2.5cm, top=1cm, bottom=1cm]{geometry}
\usepackage[osf]{mathpazo}
\signature{Thomas Guillerme \\ Natalie Cooper}
\address{Zoology building \\ Trinity College Dublin \\ Dublin 2, Ireland \\ \\ guillert@tcd.ie}
\longindentation=0pt
\begin{document}

\begin{letter}{}
\opening{Dear Editors,}

In recent years there has been growing interest in building phylogenies that contain both living and fossil taxa (e.g. Quental and Marshall 2006 TREE; Fritz et al. 2013 TREE; Heath et al. 2014 PNAS). Such phylogenies could revolutionize the way we think about macroevolutionary patterns and processes, and provide a more complete understanding of trends in biodiversity through time. Unfortunately building such trees has proved technically difficult. 

One method, the Total Evidence method, allows us to use molecular and morphological data to build phylogenies with both living and fossil species as tips (Ronquist et al. 2012 Syst Biol). This method is extremely promising because it allows us to use all the available data. However, because of the amount of data involved, the Total Evidence method is likely to be affected by missing data.

Our research article, entitled "Effects of missing data on topological inference using a Total Evidence approach", is to our knowledge, the first to thoroughly analyze the effects of missing data on tree topology in a Total Evidence framework. Using simulations ($>$ 150 CPU years worth), we find that the number of living taxa with morphological data and the overall number of morphological characters, are more important than the amount of missing data in the fossil record for recovering the "best" tree topology. Additionally, we show that Bayesian methods outperform Maximum Likelihood methods, regardless of the amount of missing data.

Our results suggest that increasing the number of taxa with morphological data and the overall number of morphological characters will greatly improve the quality of Total Evidence tree topologies. This has major implications for clades where detailed morphological data collection from living species is rare. Additionally, we recommend using a Bayesian majority consensus tree when fixing tree topology for any additional analyses.

%TG: Actually just removed this part. Checking the guidelines, we do tick all the boxes (literature, work done, results) appart from the prior interactions with PLOS.
%We feel this paper will be of particular interest to the readers of Systematic Biology as many papers investigating methods for combining living and fossil taxa and the effects of missing data on such analyses (e.g. D\'{a}valos et al. 2014; Pattinson et al. in press, Silvestro et al. 2014, Pyron 2011, Ronquist et al. 2012), have been recently published in the journal.

We look forward to hearing from you soon,

\closing{Yours sincerely,}

%Cover letter PLOS ONE guidelines
%    Concisely summarizes why your paper is a valuable addition to the scientific literature
%       %TG: line 12
%    Briefly relates your study to previously published work
%       %TG: line 14
%    Specifies the type of article you are submitting (for example, research article, systematic review, meta-analysis, clinical trial)
%       %TG: line 16
%    Describes any prior interactions with PLOS regarding the submitted manuscript
%       %TG: none
%    Suggests appropriate PLOS ONE Academic Editors to handle your manuscript (view a complete listing of our academic editors)
%       %TG: I'll go with Matthew C. Mihlbachler (who actually might have seen my talk at SVP - he was at the conference at least)
%    Lists any opposed reviewers
%       %TG: None



%%%%%%%%%%%%%%%%%%%%%%%%%%%%%%%%%%%%%%%%%%%%%%%%%%
%
%   List of editors (FOR DRAFT LETTER ONLY)
%
%%%%%%%%%%%%%%%%%%%%%%%%%%%%%%%%%%%%%%%%%%%%%%%%%%%

%TG: I'll actually go with that one - but see below for a bit more choice

%Matthew C. Mihlbachler
%NYIT College of Osteopathic Medicine
%UNITED STATES
%Expertise: Anatomy, Animal evolution, Animal musculoskeletal anatomy, Animal physiology, Biogeochemistry, Biology and life sciences, Cenozoic era, Coevolution, Comparative anatomy, Convergent evolution, Earth sciences, Ecology, Ecology and environmental sciences, Education, Evolutionary adaptation, Evolutionary biology, Evolutionary ecology, Evolutionary processes, Evolutionary systematics, Feet (anatomy), Geochemistry, Geologic time, Geology, Horns (anatomy), Joints (anatomy), Legs, Limbs (anatomy), Macroevolution, Musculoskeletal system, Organismal evolution, Paleobiology, Paleoecology, Paleontology, Parallel evolution, Phylogenetics, Science education, Social sciences, Sociology, Taphonomy, Taxonomy, Terrestrial ecology, Vertebrate paleontology, Veterinary anatomy, Veterinary science, Zoology




%%%%%%%%%%%%%%%%%%%%%%%%
%   Section Editors:
%%%%%%%%%%%%%%%%%%%%%%%%

%Corrie S. Moreau Section Editor
%Field Museum of Natural History
%UNITED STATES
%Expertise: Animal evolution, Animal phylogenetics, Biodiversity, Biogeography, Biology and life sciences, Coevolution, Entomology, Evolutionary biology, Evolutionary ecology, Evolutionary processes, Evolutionary systematics, Host-pathogen interactions, Macroevolution, Microbial ecology, Microbiology, Molecular systematics, Organismal evolution, Phylogenetics, Zoology

%Jason E Stajich Section Editor
%University of California-Riverside
%UNITED STATES
%Expertise: Biology and life sciences, Comparative genomics, Computational biology, Eukaryotic evolution, Evolutionary biology, Evolutionary developmental biology, Evolutionary processes, Evolutionary systematics, Fungal evolution, Fungal spores, Fungi, Gene duplication, Gene prediction, Gene regulation, Genetic polymorphism, Genetics, Genome analysis, Genome complexity, Genome evolution, Genome sequencing, Genomic databases, Genomics, Immune evasion, Microbial evolution, Microbiology, Model organisms, Molecular cell biology, Molecular genetics, Mycology, Natural selection, Neurospora crassa, Phylogenetics, Population genetics, Sequence analysis, Software engineering, Software tools, Transposable elements, Yeast and fungal models

%%%%%%%%%%%%%%%%%%%%%%%%
%   Paleo-ish Editors:
%%%%%%%%%%%%%%%%%%%%%%%%

%Robert Guralnick
%University of Colorado
%UNITED STATES
%Expertise: Animal phylogenetics, Animal taxonomy, Atmospheric science, Biodiversity, Biogeography, Biological data management, Biology and life sciences, Climate change, Climatology, Comparative genomics, Computational biology, Computer and information sciences, Databases, Earth sciences, Ecological metrics, Ecology and environmental sciences, Evolutionary biology, Evolutionary modeling, Evolutionary systematics, Information technology, Mammalogy, Paleontology, Phylogenetics, Population ecology, Population growth, Taxonomy, Text mining, Zoology

%Ulrich Joger
%State Natural History Museum
%GERMANY
%Expertise: Biology and life sciences, Conservation science, Evolutionary biology, Evolutionary processes, Evolutionary systematics, Paleontology, Phylogenetics, Reptile biology, Speciation, Zoology

%Ludovic Orlando
%Natural History Museum of Denmark, University of Copenhagen
%DENMARK
%Expertise: Anthropology, Biochemistry, Biodiversity, Biology and life sciences, Comparative genomics, Conservation science, DNA, DNA amplification, DNA modification, DNA repair, Ecology and environmental sciences, Evolutionary biology, Evolutionary genetics, Evolutionary processes, Evolutionary systematics, Genetic drift, Genetic polymorphism, Genetics, Genome analysis, Genome evolution, Genome sequencing, Genomics, Metagenomics, Molecular systematics, Nucleic acids, Paleoanthropology, Paleobiology, Paleoecology, Paleontology, Phylogenetics, Population genetics, Sequence assembly tools, Species extinction

%Nicholas Pyenson
%Smithsonian Institution
%UNITED STATES
%Expertise: Animal evolution, Animal phylogenetics, Animal taxonomy, Biodiversity, Biology and life sciences, Cenozoic era, Community assembly, Community structure, Earth sciences, Ecology and environmental sciences, Evolutionary biology, Evolutionary ecology, Evolutionary processes, Evolutionary systematics, Geologic time, Geology, Macroecology, Macroevolution, Mammalogy, Marine biology, Marine ecology, Micropaleontology, Niche construction, Organismal evolution, Paleobiology, Paleoclimatology, Paleoecology, Paleontology, Phylogenetics, Sedimentary geology, Speciation, Species extinction, Species interactions, Taphonomy, Taxonomy, Vertebrate paleontology, Zoology


%%%%%%%%%%%%%%%%%%%%%%%%
%   All Editors:
%%%%%%%%%%%%%%%%%%%%%%%%

%Zaid Abdo
%Institution and Department: Agricultural Research Service
%UNITED STATES
%Expertise: Applied microbiology, Artificial ecosystems, Bacterial taxonomy, Bacteriology, Bayes theorem, Biology and life sciences, Biostatistics, Biota, Biotechnology, Coastal ecology, Community assembly, Community ecology, Community structure, Computational biology, Computer and information sciences, Computer modeling, Ecology and environmental sciences, Ecosystem functioning, Ecosystem modeling, Ecosystems, Environmental biotechnology, Evolutionary adaptation, Evolutionary biology, Evolutionary genetics, Evolutionary modeling, Evolutionary processes, Evolutionary systematics, Marine biology, Marine environments, Mathematics, Microbial ecology, Microbiology, Mutation, Parallel evolution, Phylogenetics, Probability theory, Software design, Software engineering, Spatial and landscape ecology, Statistical methods, Statistics (mathematics), Systems ecology

%Alessandro Achilli
%University of Perugia
%ITALY
%Expertise: Biology and life sciences, Evolutionary biology, Evolutionary genetics, Gene flow, Genetic polymorphism, Genetics, Genome evolution, Haplotypes, Human evolution, Human genetics, Mitochondrial diseases, Molecular evolution, Organismal evolution, Phylogenetics, Physical anthropology, Population genetics

%Rodney Adam
%Aga Khan University Hospital Nairobi
%KENYA
%Expertise: Bacterial diseases, Biochemistry, Biology and life sciences, Computational biology, Epidemiology, Evolutionary biology, Evolutionary systematics, Gastrointestinal infections, Gene expression, Genetics, Genome sequencing, Genomics, Giardiasis, Global health, Infectious disease epidemiology, Infectious diseases, Marine biology, Marine conservation, Medicine and health sciences, Meningitis, Microbiology, Model organisms, Neglected tropical diseases, Nucleic acids, Parasitic diseases, Parasitology, Phylogenetics, Population biology, Population genetics, Protein translation, Protozoan infections, Protozoan models, RNA

%Alexander V. Alekseyenko
%New York University School of Medicine
%UNITED STATES
%Expertise: Algorithms, Bayes theorem, Biology and life sciences, Biostatistics, Computer and information sciences, Computer applications, Computer modeling, Effective population size, Evolutionary biology, Evolutionary ecology, Evolutionary genetics, Evolutionary processes, Evolutionary systematics, Genetics, Genome analysis, Genome evolution, Genome-wide association studies, Genomics, Mathematical computing, Mathematics, Microbial ecology, Microbial evolution, Microbial mutation, Microbiology, Molecular systematics, Natural selection, Neutral theory, Phylogenetics, Population genetics, Probability theory, Sequence analysis, Software tools, Statistical methods, Statistics (mathematics), Stochastic processes

%Robin Allaby
%University of Warwick
%UNITED KINGDOM
%Expertise: Agroecology, Biogeography, Biology and life sciences, Comparative genomics, Computational biology, Conservation science, Ecological economics, Ecology and environmental sciences, Evolutionary adaptation, Evolutionary biology, Evolutionary ecology, Evolutionary genetics, Evolutionary modeling, Evolutionary processes, Evolutionary systematics, Evolutionary theory, Flowering plants, Genetic drift, Genetics, Genome evolution, Genomics, Hybridization, Introgression, Metagenomics, Molecular genetics, Molecular systematics, Mutation, Natural selection, Paleobiology, Paleobotany, Paleoecology, Paleontology, Phyletic patterns, Phylogenetics, Plant evolution, Plant genetics, Plant genomics, Plant phylogenetics, Plant science, Plants, Population biology, Population genetics, Population modeling

%Nadir Alvarez
%University of Lausanne
%SWITZERLAND
%Expertise: Biogeography, Coevolution, Evolutionary biology, Evolutionary ecology, Evolutionary systematics, Molecular systematics, Phylogenetics, Trophic interactions

%Maria Anisimova
%Swiss Federal Institute of Technology (ETH Zurich)
%SWITZERLAND
%Expertise: Biochemistry, Biology and life sciences, Candida albicans, Comparative genomics, Computational biology, Computer and information sciences, Evolutionary biology, Evolutionary processes, Fungal classification, Fungal evolution, Genome analysis, Genome evolution, Genomics, Kluyveromyces lactis, Microbial taxonomy, Microbiology, Molecular systematics, Mycology, Natural selection, Neutral theory, Photosynthesis, Phylogenetics, Plant genetics, Plant physiology, Plant science, Population genetics, Protein interactions, Protein structure, Proteins, Proteomics, Saccharomyces cerevisiae, Taxonomy, Transcriptome analysis, Yeast, Yeast and fungal models

%Wolfgang Arthofer
%University of Innsbruck
%AUSTRIA
%Expertise: Animal evolution, Animal genetics, Animal phylogenetics, Biogeography, Biology and life sciences, Computational biology, Ecology, Ecology and environmental sciences, Ecosystem modeling, Evolutionary biology, Evolutionary ecology, Evolutionary genetics, Evolutionary modeling, Evolutionary systematics, Gene flow, Genetic drift, Genetics, Genome sequencing, Genomics, Haplotypes, Host-pathogen interactions, Microbial ecology, Microbiology, Molecular genetics, Mutation, Natural selection, Neutral theory, Organismal evolution, Phylogenetics, Plant ecology, Plant evolution, Plant phylogenetics, Plant science, Ploidy, Population biology, Population ecology, Population genetics, Sequence analysis, Terrestrial ecology

%Jonathan H. Badger
%J. Craig Venter Institute
%UNITED STATES
%Expertise: Algae, Bacterial evolution, Bacterial taxonomy, Bacteriology, Biology and life sciences, Comparative genomics, Computational biology, Ecology and environmental sciences, Evolutionary biology, Evolutionary systematics, Genome evolution, Genome sequencing, Genomics, Marine biology, Metagenomics, Microbial ecology, Microbial evolution, Microbial taxonomy, Microbiology, Molecular systematics, Phylogenetics, Sequence analysis, Software engineering, Software tools, Taxonomy

%Stephanie Bertrand
%Laboratoire Oceanologique de Banyuls sur Mer
%FRANCE
%Expertise: Animal genetics, Animal models, Biology and life sciences, Biotechnology, Developmental biology, Evolutionary biology, Evolutionary developmental biology, Gene expression, Gene function, Gene identification and analysis, Genetics, Genome evolution, Hormone receptor signaling, Hormones, Membrane receptor signaling, Model organisms, Molecular cell biology, Molecular genetics, Phylogenetics, Signal transduction, Zoology

%Jeffrey L Blanchard
%University of Massachusetts
%UNITED STATES
%Expertise: Agriculture, Animal models, Applied microbiology, Bacterial pathogens, Bacteriology, Biocatalysis, Biochemistry, Biofuels, Biology and life sciences, Biotechnology, Carbohydrate metabolism, Computational biology, Ecology and environmental sciences, Enzymes, Escherichia coli, Evolutionary biology, Evolutionary systematics, Gene ontologies, Genetic engineering, Genetic networks, Genetics, Genome analysis, Genomics, Metabolism, Microarrays, Microbial ecology, Microbial evolution, Microbial metabolism, Microbial mutation, Microbiology, Model organisms, Molecular cell biology, Mouse models, Nucleic acids, Phylogenetics, Signaling networks, Soil ecology, Systems biology

%Bazartseren Boldgiv
%National University of Mongolia
%MONGOLIA
%Expertise: Atmospheric science, Bibliometrics, Biodiversity, Biology and life sciences, Biomass (ecology), Climate change, Climatology, Community ecology, Community structure, Conservation science, Earth sciences, Ecological metrics, Ecology and environmental sciences, Ecosystem functioning, Ecosystems, Environmental protection, Evolutionary adaptation, Evolutionary biology, Evolutionary ecology, Evolutionary processes, Evolutionary systematics, Freshwater ecology, Freshwater environments, Global change ecology, Phylogenetics, Plant ecology, Plant science, Population biology, Population dynamics, Population ecology, Population growth, Productivity (ecology), Research assessment, Science policy, Soil ecology, Species interactions, Terrestrial environments, Theoretical ecology

%Sheila Mary Bowyer
%University of Pretoria/ NHLS TAD
%SOUTH AFRICA
%Expertise: Computational biology, Evolutionary genetics, Genetic epidemiology, Hepatitis, Hepatitis B, Hepatitis C, Infectious disease epidemiology, Infectious hepatitis, Microbial evolution, Molecular epidemiology, Phylogenetics, Viral diseases, Viral evolution, Virology

%Laura M. Boykin
%The University of Western Australia
%AUSTRALIA
%Expertise: Biology and life sciences, Evolutionary biology, Evolutionary systematics, Phylogenetics, Taxonomy

%Sean Brady
%Smithsonian National Museum of Natural History
%UNITED STATES
%Expertise: Animal behavior, Animal evolution, Animal phylogenetics, Animal taxonomy, Biology and life sciences, Cladistics, Developmental biology, Entomology, Evolutionary biology, Evolutionary genetics, Evolutionary systematics, Molecular systematics, Organism development, Organismal evolution, Phylogenetics, Taxonomy, Zoology

%Bluma Brenner
%Lady Davis Institute
%CANADA
%Expertise: Biochemistry, Biology and life sciences, Biotechnology, Clinical medicine, DNA, DNA recombination, Drug discovery, Drug research and development, Epidemiology, Evolutionary biology, Evolutionary systematics, Genetic polymorphism, Genetics, HIV, HIV epidemiology, HIV immunopathogenesis, HIV prevention, Infectious disease epidemiology, Infectious diseases, Medicine and health sciences, Microbiology, Nucleic acids, Phylogenetics, Population genetics, Sexually transmitted diseases, Viral diseases, Viral transmission and infection, Virology

%Celine Brochier-Armanet
%Université Claude Bernard - Lyon 1
%FRANCE
%Expertise: Archaeal evolution, Archaeal taxonomy, Archaean biology, Bacterial evolution, Bacterial taxonomy, Biology and life sciences, Comparative genomics, Evolutionary biology, Evolutionary systematics, Gene prediction, Genome analysis, Genome evolution, Genomic databases, Genomics, Microbial evolution, Microbiology, Molecular systematics, Organismal evolution, Phylogenetics

%Terry Brown
%Manchester Institute of Biotechnology
%UNITED KINGDOM
%Expertise: Agriculture, Archaeological excavation, Archaeometry, Barley, Biodiversity, Biology and life sciences, Cereal crops, Computational biology, Crops, Ecology and environmental sciences, Ecosystem modeling, Ecosystems, Environmental protection, Evolutionary biology, Evolutionary ecology, Evolutionary genetics, Evolutionary modeling, Evolutionary systematics, Genome analysis, Genome evolution, Genome sequencing, History of tuberculosis, Maize, Metagenomics, Molecular genetics, Paleoanthropology, Paleoecology, Paleontology, Phylogenetics, Physical anthropology, Plant ecology, Plant evolution, Plant genetics, Plant genomics, Plant phylogenetics, Plant science, Population ecology, Population genetics, Population modeling, Wheat

%Sven Buerki
%Royal Botanic Gardens, Kew
%UNITED KINGDOM
%Expertise: Biogeography, Biology and life sciences, Ecology and environmental sciences, Evolutionary biology, Evolutionary systematics, Flowering plants, Marine and aquatic sciences, Mediterranean Sea, Molecular systematics, Oceans, Phylogenetics, Plant evolution, Plant phylogenetics, Plant science, Plants

%Francesc Calafell
%Universitat Pompeu Fabra
%SPAIN
%Expertise: Biochemistry, Biology and life sciences, Biomarkers, Comparative genomics, DNA, DNA recombination, Effective population size, Evolutionary biology, Evolutionary genetics, Gene flow, Gene pool, Genetic drift, Genetic polymorphism, Genetic screens, Genetics, Genome analysis, Genome scans, Genomics, Haplotypes, Heterosis, Human genetics, Mutation, Natural selection, Nucleic acids, Paleobiology, Phenotypes, Phylogenetics, Physical anthropology, Population biology, Population genetics, Y-linked traits

%Daniele Canestrelli
%Tuscia University
%ITALY
%Expertise: Animal phylogenetics, Biodiversity, Biogeography, Biology and life sciences, Coevolution, Comparative genomics, Conservation science, Ecology and environmental sciences, Entomology, Evolutionary biology, Evolutionary ecology, Evolutionary processes, Evolutionary systematics, Mammalogy, Microevolution, Molecular systematics, Phylogenetics, Phylogeography, Population ecology, Population genetics, Reptile biology, Speciation, Zoology

%Jean K Carr
%St. James School of Medicine
%ANGUILLA
%Expertise: Biology and life sciences, Epidemiology, Global health, HIV, HIV epidemiology, Infectious disease epidemiology, Infectious diseases, Medicine and health sciences, Microbiology, Phylogenetics, Population biology, Public and occupational health, Retroviruses, Viral classification, Viral diseases, Virology

%Riccardo Castiglia
%Universita degli Studi di Roma La Sapienza
%ITALY
%Expertise: Animal phylogenetics, Animal taxonomy, Biogeography, Biology and life sciences, Cytogenetics, Evolutionary biology, Evolutionary systematics, Genetics, Karyotypes, Phylogenetics, Phylogeography, Population genetics, Taxonomy, Zoology

%Jose Castresana
%Institute of Evolutionary Biology (CSIC-UPF)
%SPAIN
%Expertise: Animal phylogenetics, Animal taxonomy, Biology and life sciences, Evolutionary biology, Evolutionary genetics, Evolutionary processes, Evolutionary systematics, Genome evolution, Mammalogy, Molecular systematics, Phylogenetics, Population genetics, Speciation, Zoology

%Nico Cellinese
%University of Florida
%UNITED STATES
%Expertise: Biogeography, Biological data management, Biology and life sciences, Computational biology, Computer and information sciences, Computer applications, Databases, Ecology and environmental sciences, Evolutionary biology, Evolutionary systematics, Flowering plants, Information technology, Molecular systematics, Phylogenetics, Plant evolution, Plant phylogenetics, Plant science, Plant taxonomy, Plants, Software design, Software engineering, Software tools, Taxonomy, Vascular plants, Web-based applications

%Gyaneshwer Chaubey
%Estonian Biocentre
%ESTONIA
%Expertise: Animal genetics, Anthropology, Biology and life sciences, Computational biology, Cultural anthropology, Ecology and environmental sciences, Effective population size, Evolutionary biology, Evolutionary genetics, Evolutionary modeling, Evolutionary systematics, Gene flow, Gene pool, Genetic drift, Genetic polymorphism, Genetic screens, Genetics, Genome sequencing, Genome-wide association studies, Genomics, Haplotypes, Heredity, Human genetics, Medicine and health sciences, Microarrays, Mutation, Natural selection, Nutrition, Obesity, Phylogenetics, Physical anthropology, Population biology, Population ecology, Population genetics, Population modeling, Sequence analysis, Social anthropology, Social sciences

%Shilin Chen
%Chinese Academy of Medical Sciences, Peking Union Medical College
%CHINA
%Expertise: Algae, Biochemistry, Biological data management, Biological transport, Biosynthesis, Biotechnology, Chloroplasts, Comparative genomics, Computational biology, Databases, Evolutionary biology, Evolutionary systematics, Fungal biochemistry, Fungi, Genetic engineering, Genetically modified plants, Genome analysis, Genome evolution, Genome expression analysis, Genome sequencing, Genome-wide association studies, Genomics, Metabolism, Microbiology, Molecular systematics, Mycology, Phycology, Phylogenetics, Plant biotechnology, Plant cell biology, Plant evolution, Plant microbiology, Plant science, RNA structure, Sequence analysis, Sequence assembly tools, Transcriptome analysis, Transgenic engineering

%Xiao-Yong Chen
%East China Normal University
%CHINA
%Expertise: Biodiversity, Biogeography, Biology and life sciences, Coevolution, Community ecology, Conservation science, Ecology and environmental sciences, Ecosystems, Effective population size, Evolutionary biology, Evolutionary genetics, Evolutionary processes, Evolutionary systematics, Gene flow, Genetic drift, Genetic polymorphism, Genetics, Haplotypes, Phylogenetics, Plant ecology, Plant genetics, Plant phylogenetics, Plant science, Plant-environment interactions, Plants, Population ecology, Population genetics, Trophic interactions, Vascular plants

%Tzen-Yuh Chiang
%National Cheng-Kung University
%TAIWAN
%Expertise: Biodiversity, Biogeography, Biology and life sciences, Computational biology, Ecology and environmental sciences, Ecosystems, Evolutionary biology, Evolutionary ecology, Evolutionary genetics, Evolutionary systematics, Genetics, Genome evolution, Genome sequencing, Genomics, Haplotypes, Metagenomics, Phylogenetics, Plant ecology, Plant evolution, Plant genetics, Plant genomics, Plant phylogenetics, Plant science, Population biology, Population genetics

%Anuradha Chowdhary
%V.P.Chest Institute
%INDIA
%Expertise: Aspergillosis, Biology and life sciences, Candidiasis, Computational biology, Epidemiology, Evolutionary biology, Evolutionary systematics, Fungal diseases, Fungi, Infectious disease epidemiology, Infectious diseases, Medical microbiology, Medicine and health sciences, Microbiology, Mycology, Neonatology, Pediatrics, Phylogenetics, Population biology, Respiratory system, Sequence analysis, Sexually transmitted diseases, Yeast

%Roberta Cimmaruta
%Tuscia University
%ITALY
%Expertise: Amphibians, Animal phylogenetics, Animals, Aquatic environments, Biodiversity, Biogeography, Biology and life sciences, Cladistics, Conservation genetics, DNA barcoding, Direct sequencing, Ecology, Ecology and environmental sciences, Evolutionary biology, Evolutionary ecology, Evolutionary systematics, Fisheries science, Fishes, Frogs, Marine biology, Marine fish, Molecular biology, Molecular biology techniques, Molecular systematics, Natural resources, Organisms, Parasite evolution, Parasitology, Phyletic patterns, Phylogenetics, Phylogeography, Population genetics, Salamanders, Sequencing techniques, Taxonomy, Terrestrial environments, Toads, Vertebrates

%Donald James Colgan
%Australian Museum
%AUSTRALIA
%Expertise: Animal evolution, Animal genetics, Animal phylogenetics, Animal taxonomy, Cladistics, DNA, Effective population size, Evolutionary biology, Evolutionary systematics, Gene flow, Genomics, Malacology, Microevolution, Molecular systematics, Nucleic acids, Organismal evolution, Phylogenetics, RNA, Taxonomy, Zoology

%Richard Cordaux
%University of Poitiers
%FRANCE
%Expertise: Bacterial evolution, Bacteriology, Biology and life sciences, Comparative genomics, Computational biology, Evolutionary biology, Evolutionary genetics, Evolutionary systematics, Genome evolution, Microbial evolution, Microbiology, Molecular cell biology, Phylogenetics, Population genetics, Retrotransposons, Transposable elements

%Richard Culleton
%Institute of Tropical Medicine
%JAPAN
%Expertise: Biology and life sciences, Computational biology, Evolutionary biology, Evolutionary systematics, Genetics, Genome evolution, Genomics, Infectious diseases, Malaria, Medicine and health sciences, Microbiology, Parasite evolution, Parasitic diseases, Parasitic protozoans, Parasitology, Phylogenetics, Plasmodium falciparum, Plasmodium vivax, Population genetics, Protozoology, Tropical diseases, Veterinary diseases, Veterinary parasitology, Veterinary science

%Robert J Deschenes
%College of Medicine, University of South Florida
%UNITED STATES
%Expertise: Biocatalysis, Biochemistry, Biology and life sciences, Biotechnology, Cell membranes, Cell signaling, Cytochemistry, Enzymes, Epigenetics, Evolutionary biology, Evolutionary systematics, Genetics, MAPK signaling cascades, Membrane proteins, Model organisms, Molecular cell biology, Oncogenic signaling, Phylogenetics, Protein interactions, Proteins, Saccharomyces cerevisiae, Signal transduction, Signaling cascades, Stress signaling cascade, Transferases, Transmembrane proteins, Yeast and fungal models

%Sebastien Duperron
%Universite Pierre et Marie Curie
%FRANCE
%Expertise: Animal taxonomy, Animals, Behavioral ecology, Biodiversity, Biogeochemistry, Biogeography, Biology and life sciences, Bivalves, Community ecology, Ecology, Ecology and environmental sciences, Ecosystems, Energy flow, Evolutionary biology, Evolutionary ecology, Evolutionary systematics, Gametogenesis, Invertebrates, Marine and aquatic sciences, Marine biology, Marine ecology, Microbial taxonomy, Microbiology, Molecular systematics, Molluscs, Mussels, Oceanography, Oceans, Organisms, Pacific Ocean, Phylogenetics, Physiological processes, Physiology, Sequence analysis, Sexual reproduction, Species interactions, Symbiosis, Taxonomy, Trophic interactions, Zoology

%Susanna Esposito
%Fondazione IRCCS Ca' Granda Ospedale Maggiore Policlinico, Università degli Studi di Milano
%ITALY
%Expertise: Autoimmune diseases, Bacterial diseases, Biology and life sciences, Child health, Clinical immunology, Evolutionary biology, Evolutionary systematics, Genome evolution, Genomics, Immune response, Immunity, Infectious diseases, Kawasaki disease, Medicine and health sciences, Meningococcal disease, Microbiology, Pediatric otolaryngology, Pediatric pulmonology, Pediatrics, Phylogenetics, Pneumococcus, Preventive medicine, Public and occupational health, Pulmonology, Respiratory infections, Rhinovirus infection, Travel-associated diseases, Vaccination and immunization, Vaccine development, Viral diseases, Viral load, Viral transmission and infection, Virology

%Matthew W Fields
%Montana State Univeristy
%UNITED STATES
%Expertise: Algae, Archaean biology, Bacterial biofilms, Bacteriology, Bioenergy, Biology and life sciences, Biotechnology, Community ecology, Computational biology, Earth sciences, Ecology and environmental sciences, Energy and power, Environmental biotechnology, Evolutionary biology, Evolutionary systematics, Gene regulation, Genetics, Genome analysis, Genomics, Metabolic networks, Microbiology, Molecular genetics, Phylogenetics, Plant science, Plants, Proteomics, Sequence analysis, Transcriptome analysis

%Paul V. A. Fine
%Berkeley
%UNITED STATES
%Expertise: Biodiversity, Biogeography, Biology and life sciences, Chemical ecology, Coevolution, Community assembly, Community ecology, Ecology and environmental sciences, Evolutionary biology, Evolutionary ecology, Evolutionary processes, Evolutionary systematics, Macroecology, Paleoecology, Phylogenetics, Plant ecology, Plant evolution, Plant phylogenetics, Soil ecology, Speciation, Trees

%Jorge Luis Folch-Mallol
%Universidad Autónoma del estado de Morelos
%MEXICO
%Expertise: Agricultural biotechnology, Agriculture, Agrochemicals, Applied microbiology, Bioalcohols, Biochemistry, Biodegradation, Biodiesel, Biofuels, Biogas, Biology and life sciences, Biomaterials, Bioremediation, Biotechnology, Cellular stress responses, Crop waste, DNA transcription, Environmental biotechnology, Evolutionary biology, Fungal biochemistry, Fungi, Gene expression, Genetic engineering, Genetically modified organisms, Genetically modified plants, Host-pathogen interactions, Metabolism, Microbiology, Model organisms, Molecular cell biology, Mycology, Nitrogen metabolism, Pest control, Pesticides, Phylogenetics, Plant biotechnology, Plant cell biology, Plant cell walls, Plant microbiology, Plant science, Transgenic engineering, Yeast, Yeast and fungal models

%Naomi Forrester
%University of Texas Medical Branch
%UNITED STATES
%Expertise: Animal models, Animal models of infection, Biology and life sciences, Calicivirus infection, Clinical immunology, Computational biology, Crimean-Congo hemorrhagic fever, Epidemiology, Evolutionary biology, Evolutionary genetics, Evolutionary systematics, Genome sequencing, Genomics, Host-pathogen interactions, Immunity, Infectious disease epidemiology, Infectious diseases, Medicine and health sciences, Microbial evolution, Microbiology, Model organisms, Molecular epidemiology, Mouse models, Organismal evolution, Phylogenetics, Population biology, Population genetics, RNA viruses, Vaccination and immunization, Vaccine development, Venezuelan equine encephalitis, Veterinary diseases, Veterinary science, Veterinary virology, Viral classification, Viral diseases, Viral vaccines, Virology, Virulence factors, Yellow fever, Zoonoses

%Robert J Forster
%Agriculture and Agri-Food Canada
%CANADA
%Expertise: Agricultural biotechnology, Agricultural production, Agriculture, Animal management, Applied microbiology, Archaeal taxonomy, Archaean biology, Bacterial taxonomy, Bacteriology, Biofuels, Biology and life sciences, Biomass (ecology), Biotechnology, Community assembly, Community ecology, Computational biology, Ecological metrics, Ecology and environmental sciences, Environmental biotechnology, Evolutionary systematics, Functional genomics, Fungal classification, Fungi, Gene expression, Genetically modified organisms, Genome analysis, Genomics, Metagenomics, Microbial ecology, Microbial taxonomy, Microbiology, Molecular genetics, Mycology, Phylogenetics, Protozoology, Species interactions, Sustainable agriculture, Taxonomy, Transcriptome analysis

%Pina Fratamico
%USDA-ARS-ERRC
%UNITED STATES
%Expertise: Agriculture, Applied microbiology, Bacterial and foodborne illness, Bacterial diseases, Bacterial pathogens, Biology and life sciences, Clinical microbiology, Clinical pathology, Comparative genomics, Diagnostic medicine, Ecology and environmental sciences, Emerging infectious diseases, Epidemiology, Escherichia coli, Evolutionary biology, Evolutionary systematics, Foodborne diseases, Foodborne organisms, Gastrointestinal infections, Genetics, Genetics of disease, Genome sequencing, Genomics, Host-pathogen interactions, Infectious diseases, Medical microbiology, Microbial control, Microbial ecology, Microbial evolution, Microbial pathogens, Microbiology, Molecular epidemiology, Pathogenesis, Pathogens, Pathology and laboratory medicine, Phylogenetics, Terrestrial environments

%Tadashi Fukami
%Stanford University
%UNITED STATES
%Expertise: Biodiversity, Biology and life sciences, Biota, Community assembly, Community ecology, Community structure, Computational biology, Ecology and environmental sciences, Ecosystem modeling, Ecosystems, Evolutionary biology, Evolutionary systematics, Flowers, Fungi, Microbiology, Mycology, Phylogenetics, Plant ecology, Plant science, Plants, Population biology, Population dynamics, Predator-prey dynamics, Soil ecology, Species interactions, Yeast

%Sudhindra R. Gadagkar
%Midwestern University
%UNITED STATES
%Expertise: Biodiversity, Biology and life sciences, Comparative genomics, Computational biology, Evolutionary genetics, Evolutionary processes, Evolutionary systematics, Functional genomics, Gene duplication, Gene flow, Gene identification and analysis, Gene prediction, Genome evolution, Genomics, Human evolution, Molecular genetics, Molecular systematics, Mutation, Natural selection, Neutral theory, Origin of life, Phylogenetics, Population genetics, Sequence analysis, Sexual selection, Speciation, Structural genomics

%Mark Gijzen
%Agriculture and Agri-Food Canada
%CANADA
%Expertise: Agriculture, Biology and life sciences, Cloning, Computational biology, Crop diseases, Crops, Epigenetics, Evolutionary biology, Evolutionary systematics, Gene expression, Gene mapping, Genetic polymorphism, Genome analysis, Genome evolution, Genome expression analysis, Genomic databases, Genomics, Host-pathogen interactions, Immune response, Immunology, Microbial metabolism, Microbiology, Molecular genetics, Mycology, Organismal cloning, Pathogenesis, Phylogenetics, Plant biochemistry, Plant genetics, Plant genomics, Plant microbiology, Plant pathogens, Plant pathology, Plant science, Plants, Population genetics, Proteins, Secondary metabolism, Seeds, Sequence databases, Toxic agents, Toxicology, Transcriptome analysis, Trees, Virulence factors

%Simonetta Gribaldo
%Institut Pasteur
%FRANCE
%Expertise: Archaeal evolution, Archaeal taxonomy, Archaean biology, Bacterial evolution, Bacteriology, Biology and life sciences, Comparative genomics, Computational biology, Evolutionary biology, Evolutionary systematics, Evolutionary theory, Genome evolution, Genomics, Microbial evolution, Microbiology, Organismal evolution, Phylogenetics, Viral evolution, Virology

%Robert Guralnick
%University of Colorado
%UNITED STATES
%Expertise: Animal phylogenetics, Animal taxonomy, Atmospheric science, Biodiversity, Biogeography, Biological data management, Biology and life sciences, Climate change, Climatology, Comparative genomics, Computational biology, Computer and information sciences, Databases, Earth sciences, Ecological metrics, Ecology and environmental sciences, Evolutionary biology, Evolutionary modeling, Evolutionary systematics, Information technology, Mammalogy, Paleontology, Phylogenetics, Population ecology, Population growth, Taxonomy, Text mining, Zoology

%Wolfgang R. Hess
%University of Freiburg
%GERMANY
%Expertise: Biology and life sciences, Coastal ecology, Comparative genomics, Computational biology, Ecological metrics, Ecology and environmental sciences, Evolutionary adaptation, Evolutionary biology, Evolutionary processes, Evolutionary systematics, Gene expression, Gene regulation, Genetics, Genome analysis, Genomics, Metagenomics, Microbial ecology, Microbial evolution, Microbiology, Molecular genetics, Photosynthetic efficiency, Phylogenetics, RNA stability, Regulatory networks, Sequence analysis, Systems biology, Transcriptome analysis

%Simon Ho
%University of Sydney
%AUSTRALIA
%Expertise: Biology and life sciences, Evolutionary biology, Evolutionary systematics, Phylogenetics

%Mandë Holford
%The City University of New York-Graduate Center
%UNITED STATES
%Expertise: Biochemistry, Biology and life sciences, Cellular neuroscience, Chemical biology, Chemistry, Clinical medicine, Evolutionary biology, Evolutionary genetics, Evolutionary systematics, Molecular neuroscience, Neuroscience, Neurotoxicology, Organismal evolution, Phylogenetics, Protein structure, Proteins, Research funding, Science education, Science policy, Science policy and economics, Toxicology, Toxin binding

%Matthias Horn
%University of Vienna
%AUSTRIA
%Expertise: Bacterial evolution, Bacterial pathogens, Bacterial taxonomy, Bacteriology, Biodiversity, Biology and life sciences, Coevolution, Comparative genomics, Computational biology, Ecology and environmental sciences, Emerging infectious diseases, Evolutionary adaptation, Evolutionary biology, Evolutionary processes, Evolutionary systematics, Functional genomics, Gene expression, Genome complexity, Genome evolution, Genome expression analysis, Genome sequencing, Genomics, Host-pathogen interactions, Metagenomics, Microbial ecology, Microbial evolution, Microbial metabolism, Microbial pathogens, Microbial taxonomy, Microbiology, Molecular genetics, Molecular systematics, Organismal evolution, Phylogenetics, Proteomics, Protozoology, Sequence analysis, Taxonomy

%Dorothee Huchon
%Tel-Aviv University
%ISRAEL
%Expertise: Animal phylogenetics, Biology and life sciences, Eukaryotic evolution, Evolutionary biology, Evolutionary systematics, Genome evolution, Genome sequencing, Genomics, Molecular cell biology, Organismal evolution, Phylogenetics, Retrotransposons, Transposable elements

%André O Hudson
%Rochester Institute of Technology
%UNITED STATES
%Expertise: Algae, Applied microbiology, Bacterial biochemistry, Bacterial pathogens, Bacteriology, Biochemistry, Biology and life sciences, Biotechnology, Chemistry, Computational biology, Crystallography, Drug discovery, Enzyme structure, Enzymes, Evolutionary biology, Evolutionary systematics, Genome sequencing, Genomics, Gram negative bacteria, Medical microbiology, Microbiology, Phylogenetics, Phytochemistry, Plant microbiology, Plant science, Plants, Proteomics, Systems biology

%Ulrich Joger
%State Natural History Museum
%GERMANY
%Expertise: Biology and life sciences, Conservation science, Evolutionary biology, Evolutionary processes, Evolutionary systematics, Paleontology, Phylogenetics, Reptile biology, Speciation, Zoology

%Fiona M. Jordan
%University of Bristol
%UNITED KINGDOM
%Expertise: Anthropology, Archaeology, Behavioral ecology, Coevolution, Cross-cultural studies, Cultural anthropology, Culture, Evolutionary biology, Evolutionary processes, Geographic and national differences, Historical linguistics, Human evolution, Human families, Linguistic anthropology, Linguistic geography, Linguistics, Macroevolution, Natural language, Phylogenetics, Physical anthropology, Psychological anthropology, Semantics, Social anthropology, Social sciences, Sociology, Virtual archaeology

%Jason M. Kamilar
%Midwestern University & Arizona State University
%UNITED STATES
%Expertise: Animal behavior, Animal evolution, Animal phylogenetics, Anthropology, Behavioral ecology, Biodiversity, Biogeography, Biology and life sciences, Community assembly, Community ecology, Community structure, Conservation science, Ecological metrics, Ecology and environmental sciences, Evolutionary biology, Evolutionary ecology, Evolutionary systematics, Evolutionary theory, Extinction risk, Geography, Geoinformatics, Geostatistics, Human evolution, Macroecology, Mammalogy, Organismal evolution, Phylogenetics, Physical anthropology, Primatology, Sensory systems, Social sciences, Spatial and landscape ecology, Spatial autocorrelation, Species interactions, Terrestrial ecology, Zoology

%Manfred Kayser
%Erasmus University Medical Center
%NETHERLANDS
%Expertise: Biogeography, Demography, Forensic pathology, Genetic epidemiology, Genetic polymorphism, Genetic testing, Haplotypes, Human evolution, Human genetics, Human geography, Natural selection, Phylogenetics, Physical anthropology, Population genetics

%Valerio Ketmaier
%Institute of Biochemistry and Biology
%GERMANY
%Expertise: Agriculture, Animal phylogenetics, Aquaculture, Biodiversity, Biology and life sciences, Computational biology, Conservation science, Ecology and environmental sciences, Evolutionary biology, Evolutionary processes, Evolutionary systematics, Fish biology, Gene flow, Genetic drift, Genetic polymorphism, Haplotypes, Molecular systematics, Paleoclimatology, Paleontology, Phylogenetics, Population genetics, Reptile biology, Sequence analysis, Speciation, Zoology

%John R. Kirby
%University of Iowa
%UNITED STATES
%Expertise: Bacterial pathogens, Bacterial physiology, Biochemistry, Biology and life sciences, Biomacromolecule-ligand interactions, Biophysics, Cell motility, Computational biology, Developmental biology, Enzyme regulation, Enzymes, Evolutionary biology, Evolutionary systematics, Flagellar motility, Gene expression, Gene function, Gene identification and analysis, Gene regulation, Genetics, Microbial ecology, Microbial growth and development, Microbial mutation, Microbiology, Molecular cell biology, Molecular genetics, Mutagenesis, Mutation, Phylogenetics, Signal transduction

%Ned B. Klopfenstein
%USDA Forest Service - RMRS
%UNITED STATES
%Expertise: Agriculture, Biodiversity, Biology and life sciences, Community assembly, Community ecology, Community structure, Computational biology, Conservation science, DNA, Ecological risk, Ecology and environmental sciences, Evolutionary biology, Evolutionary ecology, Evolutionary processes, Evolutionary systematics, Forestry, Fungal classification, Fungal evolution, Genetic polymorphism, Genetics, Host-pathogen interactions, Microbiology, Molecular cell biology, Molecular genetics, Mycology, Nucleic acids, Pest control, Phylogenetics, Plant pathogens, Plant pathology, Plant science, Population biology, Population dynamics, Population ecology, Population genetics, Species interactions

%Michael Knapp
%Bangor University
%UNITED KINGDOM
%Expertise: Biogeography, Biology and life sciences, Ecology and environmental sciences, Evolutionary biology, Evolutionary genetics, Evolutionary systematics, Genetics, Molecular systematics, Paleobiology, Phylogenetics

%Sergios-Orestis Kolokotronis
%Fordham University
%UNITED STATES
%Expertise: Animal evolution, Animal phylogenetics, Biology and life sciences, Cladistics, Coevolution, Comparative genomics, Evolutionary adaptation, Evolutionary biology, Evolutionary emergence, Evolutionary genetics, Evolutionary processes, Evolutionary systematics, Genetic drift, Genetics, Genome evolution, Genomics, Hybridization, Introgression, Mammalogy, Microbial evolution, Molecular systematics, Mutation, Natural selection, Organismal evolution, Phylogenetics, Population genetics, Speciation, Species extinction, Veterinary epidemiology, Veterinary microbiology, Veterinary science, Zoology

%Anna Kramvis
%University of the Witwatersrand
%SOUTH AFRICA
%Expertise: Antivirals, Biochemistry, Biology and life sciences, Computational biology, DNA, DNA amplification, Epidemiology, Evolutionary biology, Evolutionary systematics, Gastroenterology and hepatology, Genomics, HIV, Hepatitis, Hepatitis B, Infectious disease epidemiology, Infectious diseases, Infectious hepatitis, Liver diseases, Medicine and health sciences, Microbiology, Molecular cell biology, Nucleic acids, Phylogenetics, Population biology, Sequence analysis, Sequence databases, Viral disease diagnosis, Viral diseases, Viral envelope, Viral evolution, Viral load, Viral replication, Viral transmission and infection, Virology

%Greger Larson
%University of Oxford
%UNITED KINGDOM
%Expertise: Animal evolution, Animal phylogenetics, Animal taxonomy, Biodiversity, Biogeography, Biology and life sciences, Comparative genomics, Ecology and environmental sciences, Eukaryotic evolution, Evolutionary biology, Evolutionary ecology, Evolutionary processes, Evolutionary systematics, Functional genomics, Genome sequencing, Genomics, Natural selection, Organismal evolution, Paleontology, Phenetic evolution, Phylogenetics, Population biology, Population dynamics, Species extinction, Taxonomy, Zoology

%Sebastien Lavergne
%CNRS / Université Joseph-Fourier
%FRANCE
%Expertise: Biodiversity, Biogeography, Community assembly, Community ecology, Macroecology, Macroevolution, Phylogenetics, Population genetics

%Olivier Lespinet
%Université Paris-Sud
%FRANCE
%Expertise: Biological data management, Biology and life sciences, Comparative genomics, Fungal evolution, Fungi, Gene duplication, Gene identification and analysis, Genome analysis, Genome evolution, Genome sequencing, Genomic databases, Metabolic networks, Metabolic pathways, Metagenomics, Microbial evolution, Microbial metabolism, Microbiology, Mycology, Phylogenetics, Proteomic databases, Sequence analysis, Sequencing techniques

%Zhenyu Li
%University of Kentucky
%UNITED STATES
%Expertise: AKT signaling cascade, Animal models, Biology and life sciences, Biotechnology, Blood cells, Cardiology, Cardiovascular pharmacology, Cell adhesion, Cell signaling, Cellular types, Clinical research design, Cohort studies, Comparative genomics, Computational biology, Evolutionary modeling, Gene expression, Gene identification and analysis, Genetic engineering, Genome analysis, Genome sequencing, Hematology, Integrins, MAPK signaling cascades, Medicine and health sciences, Membrane receptor signaling, Membrane trafficking, Microbiology, Model organisms, Molecular cell biology, Molecular genetics, Mouse models, Nucleotide receptor signaling, Phylogenetics, Platelets, Protein kinase C signaling, Sequence analysis, Signal transduction, Signaling cascades, Transgenic engineering, Viral classification, Virology

%Damon P. Little
%The New York Botanical Garden
%UNITED STATES
%Expertise: Biodiversity, Biological data management, Biology and life sciences, Cladistics, Comparative genomics, Computational biology, Concerted evolution, DNA amplification, Epigenetics, Evolutionary biology, Evolutionary systematics, Flowering plants, Gene ontologies, Genetics, Genome analysis, Genome evolution, Genomic databases, Genomics, Gymnosperms, Haplotypes, Microbiology, Molecular systematics, Phyletic patterns, Phylogenetics, Plant anatomy, Plant evolution, Plant genetics, Plant morphology, Plant science, Plant taxonomy, Plants, Population biology, Population genetics, Pteridology, RNA interference, Sequence analysis, Sequence databases, Speciation, Taxonomy, Terrestrial environments, Trees, Vascular plants, Viral load, Viral transmission and infection, Virology

%Peruzzi Lorenzo
%Università di Pisa
%ITALY
%Expertise: Biogeography, Biology and life sciences, Carnivorous plants, Cladistics, Cytogenetic analysis, Cytogenetics, DNA barcoding, Departures from diploidy, Dicotyledons, Ecology, Evolutionary biology, Evolutionary ecology, Evolutionary processes, Evolutionary systematics, Flowering plants, Genetics, Geobotany, Haplotypes, Hybridization, Karyotypes, Lily, Molecular systematics, Monocotyledons, Organisms, Phenetics, Phyletic patterns, Phylogenetics, Phylogeography, Plant ecology, Plant evolution, Plant phylogenetics, Plant science, Plant taxonomy, Plant-animal interactions, Plant-insect interactions, Plants, Ploidy, Pollination, Polyploidy, Population genetics, Silene, Speciation, Species delimitation, Taxonomy, Vascular plants

%Connie Lovejoy
%Laval University
%CANADA
%Expertise: Archaeal biochemistry, Archaean biology, Astrobiology, Atmospheric science, Bacterial biofilms, Bacteriology, Biodiversity, Biogeography, Biology and life sciences, Biota, Carbon cycle, Chemistry, Climate change, Climatology, Coastal ecology, Community ecology, Comparative genomics, Computational biology, Earth sciences, Ecology and environmental sciences, Ecosystems, Evolutionary biology, Evolutionary systematics, Genome analysis, Genomics, Geochemistry, Global change ecology, Marine and aquatic sciences, Marine biology, Marine monitoring, Metagenomics, Microbial ecology, Microbiology, Oceanography, Origin of life, Phylogenetics, Systems biology, Transcriptome analysis

%Helge Thorsten Lumbsch
%Field Museum of Natural History
%UNITED STATES
%Expertise: Biology and life sciences, Evolutionary biology, Evolutionary systematics, Fungal classification, Fungal evolution, Fungi, Microbial taxonomy, Microbiology, Molecular systematics, Mycology, Phylogenetics, Taxonomy

%Cecilio López-Galíndez
%Centro Nacional de Microbiología - Instituto de Salud Carlos III
%SPAIN
%Expertise: Antivirals, Biology and life sciences, Evolutionary biology, Evolutionary systematics, HIV, HIV immunopathogenesis, Host cells, Host-pathogen interactions, Infectious diseases, Medicine and health sciences, Microbiology, Phylogenetics, Viral diseases, Viral evolution, Viral transmission and infection, Virology

%Carlos López-Vaamonde
%Institut National de la Recherche Agronomique (INRA),
%FRANCE
%Expertise: Agriculture, Animal behavior, Animal evolution, Animal phylogenetics, Bayes theorem, Biogeography, Biology and life sciences, Cladistics, Coevolution, Computational biology, Conservation science, Endocrine physiology, Endocrine system, Entomology, Evolutionary biology, Evolutionary genetics, Evolutionary systematics, Genetics, Mathematics, Molecular genetics, Molecular systematics, Organismal evolution, Pest control, Pheromones, Phylogenetics, Probability theory, Sequence analysis

%Stefan Lötters
%Trier University
%GERMANY
%Expertise: Agriculture, Agrochemicals, Amphibians, Animals, Atmospheric science, Biodiversity, Biogeography, Biology and life sciences, Climate change, Climatology, Community assembly, Community ecology, Earth sciences, Ecological niches, Ecology, Ecology and environmental sciences, Ecosystems, Engineering and technology, Environmental engineering, Evolutionary biology, Evolutionary processes, Evolutionary systematics, Frogs, Geographic distribution, Habitats, Herbicides, Invasive species, Marine ecosystems, Organisms, Phylogenetics, Phylogeography, Pollution, Population biology, Population dynamics, Population genetics, Predation, Salamanders, Speciation, Species colonization, Trophic interactions, Vertebrates, Water pollution

%Keping Ma
%Institute of Botany, Chinese Academy of Sciences
%CHINA
%Expertise: Biodiversity, Biology and life sciences, Biomass (ecology), Community assembly, Community ecology, Computational biology, Conservation science, Ecological metrics, Ecology and environmental sciences, Ecosystems, Evolutionary biology, Evolutionary ecology, Evolutionary systematics, Forestry, Genome evolution, Genomics, Leaves, Microbial ecology, Microbiology, Phylogenetics, Plant ecology, Plant growth and development, Plant phylogenetics, Plant science, Plant-environment interactions, Plants, Population ecology, Soil ecology, Species interactions, Terrestrial ecology, Trees

%Jesus E. Maldonado
%Smithsonian Conservation Biology Institute
%UNITED STATES
%Expertise: Animal behavior, Animal evolution, Animal genetics, Animal phylogenetics, Animal physiology, Behavioral ecology, Biodiversity, Biogeography, Biology and life sciences, Conservation science, DNA structure, Ecological metrics, Ecology and environmental sciences, Evolutionary adaptation, Evolutionary biology, Evolutionary ecology, Evolutionary genetics, Evolutionary processes, Evolutionary systematics, Extinction risk, Gene flow, Genetic drift, Genetic polymorphism, Genetic screens, Genetics, Heredity, Mammalogy, Microevolution, Molecular genetics, Mutation, Natural selection, Organismal evolution, Phylogenetics, Population biology, Population genetics, Spatial and landscape ecology, Speciation, Terrestrial environments, Zoology

%Luis Angel Maldonado Manjarrez
%National Autonomous University of Mexico
%MEXICO
%Expertise: Applied microbiology, Bacterial biofilms, Bacterial evolution, Bacterial pathogens, Bacterial physiology, Bacterial taxonomy, Bacteriology, Biodiversity, Biology and life sciences, Computational biology, Ecology and environmental sciences, Evolutionary biology, Evolutionary systematics, Genome analysis, Genomics, Gram negative bacteria, Gram positive bacteria, Host-pathogen interactions, Immune system, Immunology, Immunopathology, Industrial microbiology, Marine biology, Marine environments, Marine technology, Microbial evolution, Microbial pathogens, Microbial taxonomy, Microbiology, Phylogenetics, Salmonella, Sequence analysis, Sequence assembly tools, Staphylococcus, Streptococcus, Taxonomy

%Gabriel AB Marais
%CNRS/University Lyon 1
%FRANCE
%Expertise: Biology and life sciences, Chromosome biology, Comparative genomics, Computational biology, Evolutionary biology, Evolutionary genetics, Evolutionary systematics, Genome evolution, Genome expression analysis, Genomics, Organismal evolution, Phylogenetics, Plant evolution, Plant genomics, Plant phylogenetics, Plant science, Sequence analysis, X chromosome inactivation

%Stefano Mariani
%School of Environment & Life Sciences, University of Salford
%UNITED KINGDOM
%Expertise: Animal phylogenetics, Bayes theorem, Biodiversity, Biology and life sciences, Coastal ecology, Computational biology, Conservation science, Divergent evolution, Ecological metrics, Ecology and environmental sciences, Effective population size, Evolutionary adaptation, Evolutionary biology, Evolutionary ecology, Evolutionary processes, Evolutionary systematics, Fish biology, Fisheries science, Genetic drift, Genetics, Marine biology, Marine conservation, Marine ecology, Mathematics, Natural selection, Phylogenetics, Population ecology, Population genetics, Probability theory, Sequence analysis, Speciation, Zoology

%Darren P. Martin
%Institute of Infectious Disease and Molecular Medicine
%SOUTH AFRICA
%Expertise: Biology and life sciences, Computational biology, DNA viruses, Emerging viral diseases, Evolutionary biology, Evolutionary emergence, Evolutionary modeling, Evolutionary processes, Evolutionary systematics, Microbiology, Natural selection, Phylogenetics, Sequence analysis, Viral classification, Viral evolution, Virology

%Andrew McDowell
%University of Ulster
%UNITED KINGDOM
%Expertise: Acne-like disorders, Bacterial biofilms, Bacterial pathogens, Bacteriology, Biology and life sciences, Biomarkers, Comparative genomics, Computational biology, Dermatology, Diagnostic medicine, Epidemiology, Evolutionary biology, Evolutionary systematics, Genome expression analysis, Genome sequencing, Genomic databases, Genomics, Host-pathogen interactions, Infectious diseases, Medical microbiology, Medicine and health sciences, Metagenomics, Microbial evolution, Microbial pathogens, Microbiology, Pathogenesis, Pathology and laboratory medicine, Phylogenetics, Population genetics, Sequence analysis, Skin infections

%Ulrich Melcher
%Oklahoma State University
%UNITED STATES
%Expertise: Agriculture, Atmospheric science, Biodiversity, Bioinformatics, Biology and life sciences, Biostatistics, Climate change, Climatology, Computational biology, Database and informatics methods, Earth sciences, Ecology, Ecology and environmental sciences, Effective population size, Evolutionary biology, Evolutionary modeling, Evolutionary processes, Evolutionary systematics, Genetic drift, Genetic fingerprinting, Genetic fingerprinting and footprinting, Genetics, Genome analysis, Genomics, Infectious disease modeling, Integrated control, Mathematics, Microbiology, Molecular biology, Molecular biology techniques, Natural selection, Next-generation sequencing, Nonlinear dynamics, Pest control, Phylogenetics, Plant ecology, Plant-microbial interactions, Population genetics, Population modeling, Research and analysis methods, Statistics (mathematics), Sustainable agriculture, Taxonomy, Transcriptome analysis, Viral classification, Viral genomics, Virology

%Axel Meyer
%University of Konstanz
%GERMANY
%Expertise: Animal behavior, Animal models, Animal phylogenetics, Biodiversity, Biogeography, Bioinformatics, Biology and life sciences, Comparative genomics, Computational biology, Database and informatics methods, Developmental biology, Ecology, Ecology and environmental sciences, Evolutionary adaptation, Evolutionary biology, Evolutionary developmental biology, Evolutionary ecology, Evolutionary genetics, Evolutionary processes, Evolutionary systematics, Fish biology, Fish physiology, Gene duplication, Gene expression, Genetics, Genome analysis, Genomic imprinting, Genomics, Habitats, Mobile genetic elements, Molecular evolution, Molecular genetics, Molecular systematics, Phylogenetics, Phylogeography, Population genetics, Research and analysis methods, Sexual selection, Speciation, Transcriptome analysis, Transposable elements, Zebrafish, Zoology

%Matthew C. Mihlbachler
%NYIT College of Osteopathic Medicine
%UNITED STATES
%Expertise: Anatomy, Animal evolution, Animal musculoskeletal anatomy, Animal physiology, Biogeochemistry, Biology and life sciences, Cenozoic era, Coevolution, Comparative anatomy, Convergent evolution, Earth sciences, Ecology, Ecology and environmental sciences, Education, Evolutionary adaptation, Evolutionary biology, Evolutionary ecology, Evolutionary processes, Evolutionary systematics, Feet (anatomy), Geochemistry, Geologic time, Geology, Horns (anatomy), Joints (anatomy), Legs, Limbs (anatomy), Macroevolution, Musculoskeletal system, Organismal evolution, Paleobiology, Paleoecology, Paleontology, Parallel evolution, Phylogenetics, Science education, Social sciences, Sociology, Taphonomy, Taxonomy, Terrestrial ecology, Vertebrate paleontology, Veterinary anatomy, Veterinary science, Zoology

%Corrie S. Moreau Section Editor
%Field Museum of Natural History
%UNITED STATES
%Expertise: Animal evolution, Animal phylogenetics, Biodiversity, Biogeography, Biology and life sciences, Coevolution, Entomology, Evolutionary biology, Evolutionary ecology, Evolutionary processes, Evolutionary systematics, Host-pathogen interactions, Macroevolution, Microbial ecology, Microbiology, Molecular systematics, Organismal evolution, Phylogenetics, Zoology

%Gabriel Moreno-Hagelsieb
%Wilfrid Laurier University
%CANADA
%Expertise: Biochemistry, Biological data management, Biology and life sciences, Community structure, Comparative genomics, Computational biology, Escherichia coli, Evolutionary modeling, Evolutionary systematics, Gene function, Gene mapping, Gene prediction, Gene regulatory networks, Genetic networks, Genetic screens, Genetics, Genome analysis, Genome evolution, Genome-wide association studies, Genomics, Metagenomics, Microbial ecology, Microbial evolution, Microbiology, Model organisms, Phylogenetics, Protein interactions, Systems biology, Taxonomy

%Olga Cristina Pastor Nunes
%University of Porto
%PORTUGAL
%Expertise: Applied microbiology, Bacteriology, Biology and life sciences, Bioremediation, Biotechnology, Ecology and environmental sciences, Environmental biotechnology, Environmental protection, Evolutionary biology, Evolutionary systematics, Microbial ecology, Microbial metabolism, Microbiology, Phylogenetics, Toxicology

%Patrick O'Grady
%University of California, Berkeley
%UNITED STATES
%Expertise: Animal evolution, Animal phylogenetics, Animal taxonomy, Biodiversity, Biogeography, Biology and life sciences, Comparative genomics, Computational biology, Ecology and environmental sciences, Entomology, Eukaryotic evolution, Evolutionary adaptation, Evolutionary biology, Evolutionary ecology, Evolutionary genetics, Evolutionary processes, Evolutionary systematics, Genome evolution, Genomics, Organismal evolution, Phylogenetics, Population biology, Population dynamics, Population genetics, Taxonomy, Zoology

%Ludovic Orlando
%Natural History Museum of Denmark, University of Copenhagen
%DENMARK
%Expertise: Anthropology, Biochemistry, Biodiversity, Biology and life sciences, Comparative genomics, Conservation science, DNA, DNA amplification, DNA modification, DNA repair, Ecology and environmental sciences, Evolutionary biology, Evolutionary genetics, Evolutionary processes, Evolutionary systematics, Genetic drift, Genetic polymorphism, Genetics, Genome analysis, Genome evolution, Genome sequencing, Genomics, Metagenomics, Molecular systematics, Nucleic acids, Paleoanthropology, Paleobiology, Paleoecology, Paleontology, Phylogenetics, Population genetics, Sequence assembly tools, Species extinction

%Elena Papaleo
%University of Copenhagen
%DENMARK
%Expertise: Biochemistry, Biology and life sciences, Biomacromolecule-ligand interactions, Biophysical simulations, Biophysics, Chemical biology, Chemical physics, Chemistry, Computational biology, Computational chemistry, Enzyme structure, Enzymes, Evolutionary biology, Evolutionary systematics, Macromolecular structure analysis, Molecular dynamics, Molecular mechanics, Phylogenetics, Protein folding, Protein interactions, Protein structure, Proteins, Proteomics, Systems biology, Theoretical biology

%Hanu Pappu
%Washington State University
%UNITED STATES
%Expertise: Agriculture, Biology and life sciences, Biotechnology, Cereal crops, Crop diseases, Crop genetics, Crops, DNA viruses, Epidemiology, Evolutionary biology, Evolutionary systematics, Gene expression, Gene function, Gene identification and analysis, Gene regulation, Genetic screens, Genetically modified plants, Genetics, Genome evolution, Genome sequencing, Genomics, Host-pathogen interactions, Medicine and health sciences, Microbiology, Molecular epidemiology, Molecular genetics, Phylogenetics, Plant biotechnology, Plant genetics, Plant genomics, Plant pathogens, Plant pathology, Plant pests, Plant science, RNA viruses, Vector biology, Vegetables, Viral classification, Viral evolution, Viral transmission and infection, Viral vectors, Virology, Virulence factors

%Dimitrios Paraskevis
%University of Athens, Medical School
%GREECE
%Expertise: AIDS, Biology and life sciences, Clinical research design, Coevolution, Cohort studies, Epidemiology, Evolutionary biology, Evolutionary processes, Evolutionary systematics, Gastroenterology and hepatology, Gene flow, Genetic epidemiology, Genetics, Genetics of disease, HIV, HIV epidemiology, Hepatitis, Hepatitis B, Hepatitis C, Immunodeficiency viruses, Infectious diseases, Medicine and health sciences, Microbiology, Mutation, Phylogenetics, Population genetics, Retroviruses, Sexually transmitted diseases, Viral diseases, Viral evolution, Viral load, Viral transmission and infection, Virology

%Aristeidis Parmakelis
%National & Kapodistrian University of Athens, Faculty of Biology
%GREECE
%Expertise: Biodiversity, Biogeography, Biology and life sciences, Ecology, Ecology and environmental sciences, Evolutionary adaptation, Evolutionary biology, Evolutionary ecology, Evolutionary genetics, Evolutionary processes, Evolutionary systematics, Genetics, Molecular systematics, Phylogenetics, Phylogeography, Speciation, Species delimitation, Terrestrial ecology, Zoology

%Peristera Paschou
%Democritus University of Thrace
%GREECE
%Expertise: Adolescent psychiatry, Cancer genetics, Child psychiatry, Clinical genetics, Comparative genomics, Computational biology, Developmental and pediatric neurology, Genetic association studies, Genetic drift, Genetic epidemiology, Genetic polymorphism, Genetics, Genome analysis, Genome evolution, Genome-wide association studies, Genomic databases, Genomic medicine, Genomics, Human genetics, Mental health and psychiatry, Molecular genetics, Neurology, Neuropsychiatric disorders, Pharmacogenomics, Phylogenetics, Population genetics, Type 2 diabetes

%Luísa Maria Sousa Mesquita Pereira
%IPATIMUP (Institute of Molecular Pathology and Immunology of the University of Porto)
%PORTUGAL
%Expertise: Anthropology, Biogeography, Biology and life sciences, Computational biology, Effective population size, Evolutionary biology, Evolutionary genetics, Evolutionary systematics, Gene flow, Gene pool, Genetics, Haplotypes, Human evolution, Human genetics, Paleoanthropology, Paleontology, Phylogenetics, Physical anthropology, Population biology, Population genetics, Population modeling, Sequence analysis, Social sciences

%Marcio Pie
%Universidade Federal do Paraná
%BRAZIL
%Expertise: Biogeography, Evolutionary systematics, Phylogenetics, Phylogeography, Population genetics

%Paul J Planet
%Columbia University
%UNITED STATES
%Expertise: Biology and life sciences, Computational biology, Evolutionary biology, Evolutionary systematics, Genetics, Genome analysis, Genome evolution, Genome sequencing, Genomics, Hepatitis, Hepatitis C, Host-pathogen interactions, Infectious diseases, Medicine and health sciences, Microarrays, Microbial evolution, Microbial pathogens, Microbiology, Organismal evolution, Pediatrics, Phylogenetics, Viral diseases, Viral evolution, Viral immune evasion, Virology

%Art F. Y. Poon
%British Columbia Centre for Excellence in HIV/AIDS
%CANADA
%Expertise: Biology and life sciences, Computational biology, Evolutionary biology, Evolutionary genetics, Evolutionary systematics, Evolutionary theory, HIV, Infectious diseases, Microbiology, Phylogenetics, Sequence analysis, Viral diseases, Viral evolution, Virology

%Peter Prentis
%Queensland University of Technology
%AUSTRALIA
%Expertise: Animal genetics, Animal phylogenetics, Animal taxonomy, Biogeography, Biology and life sciences, Cladistics, Coevolution, Ecology and environmental sciences, Evolutionary adaptation, Evolutionary biology, Evolutionary ecology, Evolutionary genetics, Evolutionary processes, Evolutionary systematics, Gene flow, Haplotypes, Introgression, Molecular systematics, Natural selection, Neutral theory, Phylogenetics, Plant evolution, Plant genetics, Plant science, Population genetics

%Nicholas Pyenson
%Smithsonian Institution
%UNITED STATES
%Expertise: Animal evolution, Animal phylogenetics, Animal taxonomy, Biodiversity, Biology and life sciences, Cenozoic era, Community assembly, Community structure, Earth sciences, Ecology and environmental sciences, Evolutionary biology, Evolutionary ecology, Evolutionary processes, Evolutionary systematics, Geologic time, Geology, Macroecology, Macroevolution, Mammalogy, Marine biology, Marine ecology, Micropaleontology, Niche construction, Organismal evolution, Paleobiology, Paleoclimatology, Paleoecology, Paleontology, Phylogenetics, Sedimentary geology, Speciation, Species extinction, Species interactions, Taphonomy, Taxonomy, Vertebrate paleontology, Zoology

%Stefanie Pöggeler
%Georg-August-University of Göttingen Institute of Microbiology & Genetics
%GERMANY
%Expertise: Biochemistry, Biology and life sciences, Cell adhesion, Cell differentiation, Cell membranes, Cell signaling, Cellular types, Computational biology, Cytochemistry, Cytoplasmic streaming, Developmental biology, Eukaryotic cells, Evolutionary biology, Evolutionary genetics, Evolutionary systematics, Extracellular matrix, Extracellular matrix adhesions, Fertilization, Functional genomics, Fungal evolution, Fungal physiology, Fungal reproduction, Fungal spores, Fungal structure, Fungi, Genetics, Genomics, Membrane proteins, Microbial growth and development, Microbiology, Model organisms, Molecular cell biology, Mutation, Mycology, Neurospora crassa, Phylogenetics, Population genetics, Protein interactions, Proteins, Saccharomyces cerevisiae, Sexual differentiation, Signal transduction, Yeast and fungal models

%D. Ashley Robinson
%University of Mississippi Medical Center
%UNITED STATES
%Expertise: Bacteremia, Bacterial diseases, Bacterial evolution, Bacterial pathogens, Bacterial taxonomy, Bacteriology, Biology and life sciences, Comparative genomics, Emerging infectious diseases, Epidemiology, Evolutionary adaptation, Evolutionary biology, Evolutionary processes, Evolutionary systematics, Genetic polymorphism, Genetics, Gram positive bacteria, Haplotypes, Infectious disease epidemiology, Infectious diseases, Medicine and health sciences, Methicillin-resistant Staphylococcus aureus, Microbial evolution, Microbial mutation, Microbiology, Molecular epidemiology, Mutation, Phylogenetics, Population biology, Population genetics, Staphylococcus, Streptococcus, Veterinary epidemiology, Veterinary microbiology, Veterinary science

%Daniel L Rock
%University of Illinois at Urbana-Champaign
%UNITED STATES
%Expertise: Biology and life sciences, Emerging viral diseases, Evolutionary biology, Evolutionary systematics, Genetics, Genome analysis, Genomics, Medical microbiology, Medicine and health sciences, Microbial pathogens, Microbiology, Organisms, Parapox virus, Pathogens, Pathology and laboratory medicine, Phylogenetics, Poxviruses, Veterinary diagnostics, Veterinary diseases, Veterinary medicine, Veterinary pathology, Veterinary science, Veterinary virology, Viral evolution, Viral pathogens, Virology, Viruses, Zoonoses

%Igor B. Rogozin
%National Center for Biotechnology Information
%UNITED STATES
%Expertise: Acquired immune system, Animal models, Antibodies, Biology and life sciences, Bioontologies, Comparative genomics, Computational biology, DNA, DNA metabolism, DNA modification, DNA repair, DNA replication, Divergent evolution, Evolutionary biology, Evolutionary processes, Evolutionary systematics, Functional genomics, Gene expression, Genetics, Genetics of the immune system, Genome evolution, Genomics, Immune response, Immunity, Innate immune system, Model organisms, Molecular cell biology, Mouse models, Mutagenesis, Mutation, Mutational hypotheses, Nucleic acids, Phylogenetics, Prokaryotic models, Speciation, Structural genomics

%Antonis Rokas
%Vanderbilt University
%UNITED STATES
%Expertise: Aspergillosis, Biochemistry, Cancer genetics, Cladistics, Comparative genomics, Computational biology, DNA-binding proteins, Epigenetics, Evolutionary biology, Evolutionary emergence, Evolutionary genetics, Evolutionary modeling, Evolutionary processes, Evolutionary systematics, Fungal evolution, Fungi, Gene function, Genetics, Genome evolution, Microbiology, Model organisms, Molecular cell biology, Molecular epidemiology, Molecular systematics, Mycology, Natural selection, Neglected tropical diseases, Oncology, Phyletic patterns, Phylogenetics, Protein interactions, Proteins

%Pierre Roques
%CEA
%FRANCE
%Expertise: Acquired immune system, Animal models, Animal models of disease, Animal models of infection, Antivirals, Arboviral infections, Biology and life sciences, Clinical research design, Co-infections, Cytokines, Dengue fever, Emerging viral diseases, Epidemiology, Evolutionary biology, Evolutionary systematics, HIV, Hepatitis, Hepatitis B, Hepatitis E, Host cells, Immune cells, Immunity, Immunodeficiency viruses, Immunology, Immunoregulation, Infectious disease epidemiology, Infectious diseases, Innate immune system, Macaque, Malaria, Medicine and health sciences, Microbiology, Model organisms, Monocytes, Mouse models, Neglected tropical diseases, Phylogenetics, Plasmodium falciparum, Preclinical models, Viral clearance, Viral disease diagnosis, Viral diseases, Viral evolution, Viral load, Viral persistence and latency, Viral replication, Viral transmission and infection, Virology, Yellow fever

%Michael A. Russello
%University of British Columbia Okanagan
%CANADA
%Expertise: Comparative genomics, Conservation genetics, Evolutionary adaptation, Evolutionary genetics, Gene flow, Genetic drift, Hybridization, Molecular systematics, Natural selection, Phylogenetics, Phylogeography, Population genetics, Signatures of natural selection, Speciation, Species delimitation

%Andrey Rzhetsky
%University of Chicago
%UNITED STATES
%Expertise: Biology and life sciences, Biostatistics, Cell signaling, Computational biology, Computer and information sciences, Disease informatics, Epidemiology, Evolutionary biology, Evolutionary systematics, Genetic networks, Genome analysis, Genomics, Mathematics, Molecular cell biology, Molecular systematics, Phylogenetics, Population biology, Regulatory networks, Sequence analysis, Signal transduction, Statistics (mathematics), Systems biology, Text mining, Theoretical biology

%Marco Salemi
%University of Florida
%UNITED STATES
%Expertise: Bacterial diseases, Biology and life sciences, Comparative genomics, Epidemiology, Evolutionary biology, Genetics, Genomics, HIV epidemiology, Hepatitis, Hepatitis C, Host-pathogen interactions, Infectious diseases, Medicine and health sciences, Methicillin-resistant Staphylococcus aureus, Microbiology, Molecular epidemiology, Phylogenetics, Population biology, Population genetics, Viral diseases, Viral evolution, Virology

%Siba K Samal
%University of Maryland
%UNITED STATES
%Expertise: Agricultural production, Agriculture, Animal influenza, Animal management, Animal types, Antivirals, Applied microbiology, Avian influenza A viruses, Biology and life sciences, Biotechnology, Clinical immunology, Emerging viral diseases, Evolutionary biology, Evolutionary systematics, Genetic polymorphism, Genetics, Genome sequencing, Genomics, Immunity, Infectious diseases, Infectious diseases of the nervous system, Influenza, Laboratory animals, Medicine and health sciences, Microbial pathogens, Microbiology, Molecular cell biology, Mouse models, Mutation, Pathogenesis, Phylogenetics, Population genetics, Public and occupational health, RNA viruses, Recombinant proteins, Small animals, Structural proteins, Transmembrane proteins, Vaccination and immunization, Vaccine development, Vaccines, Veterinary diseases, Veterinary neurology, Veterinary pathology, Veterinary science, Veterinary virology, Viral classification, Viral disease diagnosis, Viral diseases, Viral vaccines, Virology, Wildlife, Zoonoses

%Paul Sandstrom
%National HIV and Retrovirology Laboratories
%CANADA
%Expertise: AIDS, Animal models, Antivirals, Biology and life sciences, Co-infections, Emerging viral diseases, Evolutionary biology, Evolutionary systematics, HIV, HIV diagnosis and management, HIV epidemiology, HIV immunopathogenesis, HIV prevention, Hepatitis, Hepatitis C, Immunodeficiency viruses, Immunology, Infectious diseases, Macaque, Medical microbiology, Medicine and health sciences, Microbial pathogens, Microbiology, Model organisms, Phylogenetics, RNA viruses, Retroviruses, Sexually transmitted diseases, Viral classification, Viral disease diagnosis, Viral diseases, Viral evolution, Viral transmission and infection, Viral vaccines, Virology

%Indra Neil Sarkar
%University of Vermont
%UNITED STATES
%Expertise: Biological data management, Biology and life sciences, Bioontologies, Comparative genomics, Computational biology, Computer and information sciences, Data mining, Evolutionary biology, Evolutionary systematics, Functional genomics, Gene mapping, Gene ontologies, Gene prediction, Genetic networks, Genome analysis, Genome complexity, Genome evolution, Genome sequencing, Genomic databases, Genomic medicine, Genomics, Information theory, Medicine and health sciences, Metagenomics, Natural language processing, Ontologies, Phylogenetics, Public and occupational health, Semantics, Sequence analysis, Sequence databases, Social sciences, Systems biology, Text mining

%Joseph Schacherer
%University of Strasbourg
%FRANCE
%Expertise: Applied microbiology, Biology and life sciences, Biotechnology, Chromosome biology, Chromosome structure and function, Comparative genomics, Computational biology, Evolutionary biology, Evolutionary genetics, Evolutionary processes, Evolutionary systematics, Fungal classification, Fungal evolution, Fungi, Genetic polymorphism, Genetics, Genome evolution, Genome expression analysis, Genome sequencing, Genomics, Microbiology, Model organisms, Mycology, Natural selection, Phylogenetics, Ploidy, Population genetics, Saccharomyces cerevisiae, Schizosaccharomyces pombe, Speciation, Yeast, Yeast and fungal models

%Konrad Scheffler
%University of California, San Diego
%UNITED STATES
%Expertise: Bayes theorem, Biology and life sciences, Computational biology, Computer and information sciences, Computer modeling, Evolutionary biology, Evolutionary modeling, Evolutionary processes, Evolutionary systematics, Evolutionary theory, Markov models, Mathematics, Natural selection, Neutral theory, Phylogenetics, Population genetics, Probability theory

%Timothy M. Shank
%Woods Hole Oceanographic Institution (WHOI)
%UNITED STATES
%Expertise: Animal physiology, Animal taxonomy, Antarctic Ocean, Astrobiology, Biodiversity, Biogeography, Biota, Community assembly, Coral reefs, Corals, Ecology and environmental sciences, Evolutionary biology, Extremophiles, Geochemistry, Geology, Hydrothermal vents, Marine and aquatic sciences, Marine biology, Marine ecology, Marine geology, Marine technology, Microbial ecology, Microbiology, Molecular systematics, Ocean ridges, Oceans, Phylogenetics, Plate tectonics, Spreading centers, Taxonomy, Zoology

%Igor V Sharakhov
%Virginia Tech
%UNITED STATES
%Expertise: Biology and life sciences, Centromeres, Chromatin, Chromosomal inheritance, Chromosome biology, Chromosome structure and function, Comparative genomics, Computational biology, Cytogenetic analysis, Cytogenetic techniques, Cytogenetics, Cytological landmarks, Disease vectors, Drosophila melanogaster, Entomology, Epigenetics, Evolutionary adaptation, Evolutionary biology, Evolutionary genetics, Evolutionary processes, Evolutionary systematics, Gene flow, Gene mapping, Genetic polymorphism, Genetics, Genome analysis, Genome complexity, Genome evolution, Genome expression analysis, Genome sequencing, Genomic databases, Genomics, Heredity, Histone modification, Infectious diseases, Introgression, Linkage mapping, Medicine and health sciences, Meiosis, Microbiology, Mitosis, Molecular cell biology, Mosquitoes, Phylogenetics, Ploidy, Population genetics, Quantitative trait loci, Quantitative traits, Sequence databases, Signal transduction, Speciation, Telomeres, Trait locus analysis, Transposable elements, Vector biology, Viral diseases, Yellow fever

%Matthew Shawkey
%University of Akron
%UNITED STATES
%Expertise: Animal evolution, Animal phylogenetics, Avian biology, Biology and life sciences, Chemistry, Evolutionary biology, Evolutionary developmental biology, Evolutionary systematics, Feathers, Genetics, Heredity, Host-pathogen interactions, Material properties, Materials science, Microbial pathogens, Microbiology, Optical properties, Organic chemistry, Organic materials, Organismal evolution, Phenotypes, Phylogenetics, Zoology

%Julia D Sigwart
%Queen's University Belfast
%IRELAND
%Expertise: Animal phylogenetics, Fossil wood, Malacology, Marine ecology, Molluscs, Ocean temperature, Paleobiology, Paleogenetics, Paleozoology, Phylogenetics, Sense organs, Teeth

%Matthew E. Smith
%University of Florida
%UNITED STATES
%Expertise: Bayes theorem, Biogeography, Biology and life sciences, Comparative genomics, Computational biology, Ecological metrics, Ecology and environmental sciences, Evolutionary biology, Evolutionary systematics, Fungal evolution, Genome evolution, Genome sequencing, Genomics, Mathematics, Microbiology, Molecular systematics, Mycology, Phylogenetics, Plant science, Probability theory, Sequence analysis

%Aldo Solari
%University of Chile
%CHILE
%Expertise: Evolutionary biology, Evolutionary systematics, Genome evolution, Genomics, Parasite evolution, Parasitic protozoans, Parasitology, Phylogenetics, Protozoology, Trypanosoma

%Ricard V. Solé
%Santa Fe Institute
%SPAIN
%Expertise: Artificial ecosystems, Behavioral ecology, Communications, Complex systems, Computer modeling, Computerized simulations, Ecosystem modeling, Epistasis, Evolutionary developmental biology, Gene expression, Gene regulation, Gene regulatory networks, Geocomputation, Heredity, Information technology, Molecular genetics, Paleobiology, Paleontology, Phyletic patterns, Phylogenetics, Population modeling, Social networks, Sociology, Species extinction, Systems biology, Theoretical biology

%Erik Sotka
%College of Charleston
%UNITED STATES
%Expertise: Aquatic environments, Behavioral ecology, Biogeography, Biology and life sciences, Chemical ecology, Community ecology, Community structure, Coral reefs, Corals, Ecology and environmental sciences, Energy flow, Evolutionary adaptation, Evolutionary biology, Evolutionary ecology, Evolutionary processes, Evolutionary systematics, Food web structure, Gene flow, Gene pool, Genetic drift, Genetics, Marine biology, Marine ecology, Marine environments, Natural selection, Organismal evolution, Phylogenetics, Plant evolution, Population genetics, Species interactions, Trophic interactions

%Srinand Sreevatsan
%University of Minnesota
%UNITED STATES
%Expertise: Acquired immune system, Avian influenza A viruses, B cells, Bacterial diseases, Biology and life sciences, Bovine tuberculosis, Cell signaling, Comparative genomics, Cytokines, Disease ecology, Epidemiology, Evolutionary biology, Evolutionary systematics, Gene expression, Genetic polymorphism, Host-pathogen interactions, Immune activation, Immune suppression, Immune system, Immunity, Immunological signaling, Immunomodulation, Immunoregulation, Infectious disease epidemiology, Infectious diseases, Inflammation, Innate immune system, Mechanisms of signal transduction, Medicine and health sciences, Methicillin-resistant Staphylococcus aureus, Microbiology, Monocytes, Mycobacteria, Mycobacterium avium complex, Nutrition, Phylogenetics, Population genetics, Public and occupational health, T cells, Tuberculosis, Virology, Zoonoses

%Jason E Stajich Section Editor
%University of California-Riverside
%UNITED STATES
%Expertise: Biology and life sciences, Comparative genomics, Computational biology, Eukaryotic evolution, Evolutionary biology, Evolutionary developmental biology, Evolutionary processes, Evolutionary systematics, Fungal evolution, Fungal spores, Fungi, Gene duplication, Gene prediction, Gene regulation, Genetic polymorphism, Genetics, Genome analysis, Genome complexity, Genome evolution, Genome sequencing, Genomic databases, Genomics, Immune evasion, Microbial evolution, Microbiology, Model organisms, Molecular cell biology, Molecular genetics, Mycology, Natural selection, Neurospora crassa, Phylogenetics, Population genetics, Sequence analysis, Software engineering, Software tools, Transposable elements, Yeast and fungal models

%Roscoe Stanyon
%University of Florence
%ITALY
%Expertise: Animal phylogenetics, Anthropology, Biology and life sciences, Chromosome biology, Chromosome structure and function, Comparative genomics, Cytogenetic analysis, Cytogenetic techniques, Cytogenetics, Evolutionary biology, Genetics, Genome evolution, Human evolution, Mammalogy, Molecular cell biology, Phylogenetics, Physical anthropology, Sexual selection, Social sciences, Speciation

%Genlou Sun
%Saint Mary's University
%CANADA
%Expertise: Biochemistry, Biology and life sciences, Cereal crops, Comparative genomics, Convergent evolution, Crops, DNA, Ecology and environmental sciences, Evolutionary biology, Evolutionary ecology, Evolutionary genetics, Evolutionary processes, Microevolution, Molecular systematics, Natural selection, Nucleic acids, Organismal evolution, Phylogenetics, Plant evolution, Plant genetics, Plant genomics, Plant science, Wheat

%Kumarasamy Thangaraj
%Centre for Cellular and Molecular Biology
%INDIA
%Expertise: Anthropology, Archaeology, Computational biology, Evolutionary biology, Evolutionary processes, Gene flow, Genetic polymorphism, Genetics, Genome evolution, Genomics, Haplotypes, Heredity, Human evolution, Hybridization, Indigenous populations, Linguistic anthropology, Molecular genetics, Mutation, Paleontology, Phylogenetics, Population biology, Population genetics

%Mukund Thattai
%Tata Institute of Fundamental Research
%INDIA
%Expertise: Computational biology, Eukaryotic evolution, Phylogenetics, Regulatory networks, Signaling networks, Synthetic biology, Systems biology

%Yara M. Traub-Csekö
%Instituto Oswaldo Cruz, Fiocruz
%BRAZIL
%Expertise: Animal models, Biology and life sciences, Cell signaling, Computational biology, Disease vectors, Evolutionary biology, Evolutionary systematics, Gene expression, Genetics, Genome sequencing, Genomics, Immune response, Immunology, Infectious diseases, Leishmaniasis, Medicine and health sciences, Model organisms, Molecular cell biology, Molecular genetics, Mouse models, Parasitic diseases, Phylogenetics, Proteomics, RNA interference, STAT signaling, Signal transduction

%Srikanth Prasad Tripathy
%National AIDS Research Institute
%INDIA
%Expertise: AIDS, Biology and life sciences, Breast feeding, Clinical research design, Clinical trials, HIV, HIV clinical manifestations, HIV diagnosis and management, HIV epidemiology, HIV immunopathogenesis, Infectious diseases, Medicine and health sciences, Microbiology, Molecular epidemiology, Mycobacteria, Mycobacterium avium complex, Obstetrics and gynecology, Phylogenetics, Sexually transmitted diseases, Tuberculosis, Viral diseases, Viral immune evasion, Viral transmission and infection, Virology

%Tamir Tuller
%Tel Aviv University
%ISRAEL
%Expertise: Biology and life sciences, Computational biology, Evolutionary biology, Evolutionary modeling, Evolutionary systematics, Gene expression, Genetics, Molecular cell biology, Nucleic acids, Phylogenetics, Proteomic databases, Proteomics, RNA, RNA processing, Systems biology, Theoretical biology

%Nicole Valenzuela
%Iowa State University
%UNITED STATES
%Expertise: Animal evolution, Animal physiology, Biodiversity, Conservation science, Developmental biology, Ecology and environmental sciences, Evolutionary adaptation, Evolutionary biology, Evolutionary ecology, Evolutionary processes, Freshwater ecology, Gene flow, Gene pool, Genetic polymorphism, Genetics, Global change ecology, Haplotypes, Heredity, Marine environments, Natural selection, Organism development, Organismal evolution, Phenotypes, Phylogenetics, Population genetics, Reproductive physiology, Reproductive system, Sexual reproduction, Zoology

%Steven M Vamosi
%University of Calgary
%CANADA
%Expertise: Animal evolution, Animal phylogenetics, Biodiversity, Biology and life sciences, Coastal ecology, Community assembly, Community ecology, Conservation science, Ecology and environmental sciences, Evolutionary adaptation, Evolutionary biology, Evolutionary ecology, Evolutionary processes, Evolutionary systematics, Flowering plants, Freshwater ecology, Niche construction, Organismal evolution, Phylogenetics, Plant ecology, Plant phylogenetics, Plant science, Plants, Population biology, Population dynamics, Population genetics, Restoration ecology, Seeds, Speciation, Species extinction, Species interactions, Terrestrial ecology, Zoology

%Arndt von Haeseler
%Max F. Perutz Laboratories
%AUSTRIA
%Expertise: Animal phylogenetics, Biodiversity, Biology and life sciences, Comparative genomics, Computational biology, Computer and information sciences, Ecology and environmental sciences, Evolutionary biology, Evolutionary ecology, Evolutionary genetics, Evolutionary modeling, Evolutionary processes, Evolutionary systematics, Evolutionary theory, Functional genomics, Gene duplication, Gene expression, Gene identification and analysis, Genome analysis, Genome evolution, Genome expression analysis, Genome sequencing, Genomic databases, Genomics, Markov models, Mathematics, Metagenomics, Molecular genetics, Molecular systematics, Phylogenetics, Population genetics, Probability theory, RNA, RNA interference, RNA processing, RNA stability, Sequence analysis, Software engineering, Theoretical biology, Zoology

%Daniel E. Voth
%University of Arkansas for Medical Sciences
%UNITED STATES
%Expertise: Bacterial diseases, Bacterial pathogens, Bacteriology, Cellular structures and organelles, Comparative genomics, Emerging infectious diseases, Evolutionary biology, Evolutionary systematics, Genome sequencing, Genomics, Gram negative bacteria, Host-pathogen interactions, Immunology, Infectious diseases, Microbial pathogens, Microbiology, Molecular cell biology, Pathogenesis, Phylogenetics, Protein kinase signaling cascade, Q fever, Signal transduction, Signaling cascades, Virulence factors, cAMP signaling cascade

%Ross Frederick Waller
%University of Cambridge
%UNITED KINGDOM
%Expertise: Algae, Biochemistry, Biology and life sciences, Cellular structures and organelles, Chloroplasts, Computational biology, Cytochemistry, Evolutionary biology, Evolutionary systematics, Genomics, Marine biology, Microbiology, Molecular cell biology, Origin of life, Parasite evolution, Parasitic protozoans, Parasitology, Phylogenetics, Plant cell biology, Plant science, Plants, Plastids, Protozoology

%Xugao Wang
%Insititute of Applied Ecology
%CHINA
%Expertise: Animals, Arthropoda, Beetles, Biodiversity, Biology and life sciences, Community assembly, Community ecology, Community structure, Earth sciences, Ecology, Ecology and environmental sciences, Evolutionary biology, Evolutionary systematics, Forest ecology, Forestry, Geomorphology, Habitats, Insects, Invertebrates, Organisms, Phylogenetics, Plant ecology, Plant science, Plant-environment interactions, Plants, Population biology, Population dynamics, Population ecology, Shrubs, Soil science, Spatial and landscape ecology, Temperate forests, Terrestrial environments, Topography, Trees

%John J. Welch
%University of Cambridge
%UNITED KINGDOM
%Expertise: Animal models, Biology and life sciences, Drosophila melanogaster, Evolutionary adaptation, Evolutionary biology, Evolutionary processes, Genetic drift, Genetics, Genome evolution, Genomics, Model organisms, Phylogenetics, Plasmodium falciparum, Population biology, Population genetics, Schizosaccharomyces pombe, Theoretical biology, Yeast and fungal models

%Megan J. Wilson
%University of Otago
%NEW ZEALAND
%Expertise: Anatomy, Animal phylogenetics, Animals, Arthropoda, Bees, Biochemistry, Biology and life sciences, Cell fate determination, Cell signaling, DNA, Developmental biology, Drosophila, Embryonic pattern formation, Evolutionary biology, Evolutionary developmental biology, Evolutionary systematics, Gene expression, Gene function, Gene regulation, Genetics, Honey bees, Hymenoptera, Insects, Invertebrates, Limb development, Mammals, Mice, Molecular biology, Molecular biology techniques, Molecular cell biology, Molecular genetics, Morphogenesis, Morphogenic segmentation, Notch signaling, Organism development, Organisms, Organogenesis, Pattern formation, Phylogenetics, Promoter regions, RNA, RNA sequencing, Reproductive system, Rodents, Sea squirts, Sequencing techniques, Signal transduction, Testes development, Transcription activators, Vertebrates

%Patrick CY Woo
%The University of Hong Kong
%HONG KONG
%Expertise: Aspergillosis, Bacterial diseases, Bacterial pathogens, Biology and life sciences, Chromoblastomycosis, Clinical laboratories, Clinical laboratory sciences, Coccidioidomycosis, Cryptococcosis, Cutaneous mycoses, Dermatophytosis, Emerging infectious diseases, Evolutionary systematics, Fungal classification, Fungal diseases, Fungal evolution, Fungal pathogens, Genome sequencing, Genomics, Hair and nail diseases, Healthcare-associated infections, Infectious disease control, Infectious disease surveillance, Infectious diseases, Influenza, Medical microbiology, Medicine and health sciences, Microbial pathogens, Microbiology, Mycetoma, Mycology, Nocosomial disease, Nosocomial infections, Phylogenetics, Staphylococcus aureus, Subcutaneous mycoses, Systemic mycoses, Tuberculosis, Viral diseases, Viral pathogens, Yeast infections, Zoonoses, Zygomycosis

%Peng Xu
%Chinese Academy of Fishery Sciences
%CHINA
%Expertise: Agriculture, Animal breeding, Animal genetics, Animal management, Animal types, Aquaculture, Aquatic animals, Biology and life sciences, Comparative genomics, Computational biology, Evolutionary adaptation, Evolutionary biology, Evolutionary processes, Evolutionary systematics, Fish biology, Functional genomics, Gene expression, Genetics, Genome analysis, Genome evolution, Genome sequencing, Genome-wide association studies, Genomics, Linkage mapping, Molecular cell biology, Molecular genetics, Phylogenetics, Sequence analysis, Structural genomics, Transcriptome analysis, Veterinary science, Zoology

%Ruan Yuhua
%National Center for AIDS/STD Control and Prevention, China CDC
%CHINA
%Expertise: Antiretroviral therapy, Blood plasma, China, Drug therapy, HIV, HIV infections, HIV-1, Human papillomavirus infection, Men who have sex with men, Phylogenetics, T cells, Viral load

%Andrey M Yurkov
%Leibniz Institute DSMZ-German Collection of Microorganisms and Cell Cultures
%GERMANY
%Expertise: Agriculture, Ascomycetes, Basidiomycetes, Biodiversity, Biology and life sciences, Candida albicans, Community ecology, Community structure, Cryptococcus albidus, Cryptococcus gattii, Cryptococcus neoformans, Earth sciences, Ecology, Ecology and environmental sciences, Ecosystem functioning, Ecosystems, Evolutionary biology, Evolutionary systematics, Fungal classification, Fungal pathogens, Fungi, Geography, Habitats, Human geography, Land use, Microbial ecology, Microbiology, Mycology, Oomycetes, Organisms, Pedology, Phylogenetics, Plant microbiology, Population biology, Population dynamics, Soil ecology, Soil science, Trichosporon, Yeast

%Ming Zhang
%University of Georgia
%UNITED STATES
%Expertise: Biology and life sciences, Clinical immunology, Clinical research design, Cohort studies, Comparative genomics, Computational biology, Computer and information sciences, Computer applications, Computer modeling, DNA, DNA recombination, Ecology and environmental sciences, Epidemiology, Evolutionary biology, Evolutionary systematics, Genetics, Genomics, Glycoproteins, HIV, HIV epidemiology, Human papillomavirus infection, Immune response, Immunology, Infectious diseases, Influenza, Medicine and health sciences, Microbiology, Molecular cell biology, Molecular epidemiology, Nucleic acids, Phylogenetics, Sequence analysis, Sexually transmitted diseases, Text mining, Viral diseases, Viral transmission and infection, Virology

%Maarja Öpik
%University of Tartu
%ESTONIA
%Expertise: Biodiversity, Biogeography, Biology and life sciences, Community ecology, Community structure, Fungal classification, Fungi, Metagenomics, Microbial ecology, Mycology, Mycorrhiza, Phylogenetics, Plant microbiology, Plant science, Species interactions, Symbiosis

\end{letter}
\end{document}
