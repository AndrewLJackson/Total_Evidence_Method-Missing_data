\subsection{Triplets metric details ($T_{x,y}$)}
Each triplet can be written as $I_{ijk}$=(\textit{ijk})). Where $I_{ijk}$ is equal to 0 if the the two triplets (\textit{ijk}) are the same in the two trees otherwise $I_{ijk}$ is equal to 1.
For any rooted tree there are only four possible combinations per triplets: ((\textit{j},\textit{k}),\textit{i});, ((\textit{i},\textit{k}),\textit{j}); and ((\textit{i},\textit{j}),\textit{k}); and (\textit{i},\textit{j},\textit{k}); \citep{johnson1998}.
One can calculate $S_n$, the triplet distance between two trees as:
\begin{equation}
S_n = \sum_{ijk} I_{ijk}
\end{equation}
Where:
\begin{equation}
\sum_{ijk} = \binom{n}{4} = \frac{n!}{4!(n-4)!}
\end{equation}
And where n is the number of taxa in both trees (modified from \citet{critchlowthe1996}).
When all triplets across the two trees are the same, $S_n$ is equal to 0 and when all the triplets are different $S_n$ is equal to $\binom{n}{4}$.
Because the possible number of triplets per clade is a finite number, the probability of two random trees with the same n taxa to have the same triplet is:
\begin{equation}
P({I_{ijk}}=0) = \frac{1}{4}
\end{equation}
Therefore one can calculate the probability of two random trees having the same triplets: 
\begin{equation}
P({S_{n}}=0) = \sum_{ijk} P_{I_{ijk}=0}
\end{equation}
\begin{equation}
P({S_{n}}=0) = \frac{n!}{4(3!(n-3)!}
\end{equation}
And in the same way:
\begin{equation}
P({S_{n}}=1) = \frac{3n!}{4(3!(n-3)!}
\end{equation}

\subsection{RF metric details}
The RF distance (or path difference) between two trees reflects the distance between the distributions of the tips among clades in the two trees \citep{RF1981} and can be expressed as following:
\begin{equation}
RF_{x,y} = N_{x} + N_{y} - 2C_{x,y}
\end{equation}
Where $C_{x,y}$ is the number of clades in common in the two trees. 
The minimal value of \textit{C} is equal to 1 if the two trees have the same n taxa;
the maximal value in \textit{C}=\textit{n}-2.
For a fully unresolved tree (star tree) \textit{N}=1 and for a fully resolved tree (binary tree) \textit{N}=\textit{n}-2.
The minimal and maximal topological distance for \textit{n} taxa is:
\begin{equation}
RF_{min} = 1 + 1 - 2C_{x,y}
\end{equation}
And:
\begin{equation}
RF_{max} = 2(n-2)-2
\end{equation}
One can then rescale \textit{RF.scaled} by using the maximal and minimal value for any \textit{n} taxa:
\begin{equation}
RF.scaled_{x,y} = \frac{RF_{x,y}-RF_{max}}{RF_{max}}
\end{equation}
This metric is more sensitive to taxa displacement than the Triplet distance \citep{critchlowthe1996,johnson1998,wiensmissing2003} and therefore a low value will show a good clade conservation between two trees and a high value will show a bad recovery of common clades.

\subsection{Tree comparisons}
\subsubsection{Random tree comparison scaling}
We used the comparison of 1000 random trees to obtain the mean comparison value $\bar{d}_{m,n}$\textit{(rand)} for the NTS metric.
We randomly generated two sets of 1000 trees of \textit{n} taxa using the rmtree function of ape package (v3.0-11 \citet{paradisape:2004}) that generates a given number of random Yule trees.
We calculated the $\bar{d}_{m,n}$\textit{(rand)} value using an approach similar to the RPCBTC (described below) by performing 1000 random pairwise comparisons using the TreeCmp java script \citep{Bogdanowicz2012}.

\subsubsection{Random Pairwise Bayesian Tree Comparison (RPBTC)}
We assessed the power of the Random Pairwise Bayesian Tree Comparison (RPBTC) method by comparing 1000 random trees from a posterior distribution trees set to another 1000 random trees from the same posterior distribution trees set.
We repeated this 100 times independently using the same posterior distribution trees set each time resulting in 100 replicates of the same posterior distribution trees set compared 1000 times.
We used an ANOVA to test if there was no significant difference between the replicates so that the RBTC can be replicated.
We applied this protocol on a poorly resolved tree (Low Score), a resolved tree with low support value (Medium Score) and a resolved tree with high support values (High Score).
Results are available in table ~\ref{RPBTC_testing}. %link broken

\begin{table}[ht]
  \caption{Group comparison results: difference between 100 replicates using the RPBTC method} %test with TreeCmp.anova
  \centering
  \begin{tabular}{rllrrrr}
  \hline
  Tree.Type & Used.metric & Replicates & Df & F.value & p.value \\ 
  \hline
  Low Score & RF & 100.00 & 99.00 & 0.74 & 0.98 \\ 
  Low Score & Tr & 100.00 & 99.00 & 0.97 & 0.58 \\ 
  Medium Score & RF & 100.00 & 99.00 & 0.64 & 1.00 \\ 
  Medium Score & Tr & 100.00 & 99.00 & 0.45 & 1.00 \\ 
  High Score & RF & 100.00 & 99.00 & 0.20 & 1.00 \\ 
  High Score & Tr & 100.00 & 99.00 & 0.37 & 1.00 \\ 
  \hline
\end{tabular}
  \label{RPBTC_testing}
\end{table}
