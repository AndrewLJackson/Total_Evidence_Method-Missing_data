\documentclass[11pt]{letter}
\usepackage[a4paper,left=2.5cm, right=2.5cm, top=1cm, bottom=1cm]{geometry}
\usepackage[osf]{mathpazo}
\signature{Thomas Guillerme \\ Natalie Cooper}
\address{Zoology building \\ Trinity College Dublin \\ Dublin 2, Ireland \\ \\ guillert@tcd.ie}
\longindentation=0pt
\begin{document}

\begin{letter}{}
\opening{Dear Editors,}

In recent years there has been growing interest in building phylogenies that contain both living and fossil taxa (e.g. Quental and Marshall 2006 TREE; Fritz et al. 2013 TREE; Heath et al. 2014 PNAS). Such phylogenies could revolutionize the way we think about macroevolutionary patterns and processes, and provide a more complete understanding of trends in biodiversity through time. Unfortunately building such trees has proved technically difficult. 

One method, the Total Evidence method, allows us to use molecular and morphological data to build phylogenies with both living and fossil species as tips (Ronquist et al. 2012 Syst Biol). This method is extremely promising because it allows us to use all the available data. However, because of the amount of data involved, the Total Evidence method is likely to be affected by missing data.

Our research article, entitled "Effects of missing data on topological inference using a Total Evidence approach", is to our knowledge, the first to thoroughly analyze the effects of missing data on tree topology in a Total Evidence framework. Using simulations ($>$ 150 CPU years worth), we find that the number of living taxa with morphological data and the overall number of morphological characters, are more important than the amount of missing data in the fossil record for recovering the "best" tree topology. Additionally, we show that Bayesian methods outperform Maximum Likelihood methods, regardless of the amount of missing data.

Our results suggest that increasing the number of taxa with morphological data and the overall number of morphological characters will greatly improve the quality of Total Evidence tree topologies. This has major implications for clades where detailed morphological data collection from living species is rare. Additionally, we recommend using a Bayesian majority consensus tree when fixing tree topology for any additional analyses.

We would like to submit this as a research article. We have had no prior interactions with PLoS regarding this article.

As academic editors we suggest Matthew C. Mihlbachler, Patrick O'Grady and Nicholas Pyenson. We have no opposed reviewers or editors.

We look forward to hearing from you soon,

\closing{Yours sincerely,}

\end{letter}
\end{document}
